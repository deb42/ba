\documentclass[a4paper,12pt]{article}
\usepackage[latin1]{inputenc}
\usepackage[pdftex]{color,graphicx}
\usepackage[hypertexnames=false, pdfborder={0 0 0}]{hyperref} 
\usepackage[german,ngerman]{babel}
\usepackage{fancyhdr}
\usepackage{amssymb}
\usepackage{background}
\usepackage{amsmath}
\usepackage[rflt]{floatflt}
\usepackage{tabularx}
\usepackage{ausarbeitung}
\usepackage{listings}
\usepackage{color}
\usepackage[T1]{fontenc}
\usepackage{mathpazo}
\usepackage{lmodern}

\setcounter{tocdepth}{3}
\newcommand{\pfile}[1]{{\bfseries \ttfamily #1}}

%% Diese Farben werden f�r den Quelltext verwendet
\definecolor{srcblue}{rgb}{0,0,0.5}
\definecolor{srcgray}{rgb}{0.5,0.5,0.5}
\definecolor{srcred}{rgb}{0.5,0,0}

\definecolor{mygreen}{rgb}{0,0.6,0}
\definecolor{mygray}{rgb}{0.5,0.5,0.5}


\lstdefinelanguage{JavaScript}{
  keywords={typeof, new, true, false, catch, function, return, null, catch, switch, var, if, in, for, while, do, else, case, break},
  ndkeywords={class, export, boolean, throw, implements, import, this},
  identifierstyle=\color{black},
  sensitive=false,
  comment=[l]{//},
  morecomment=[s]{/*}{*/},
  morestring=[b]',
  morestring=[b]"
}

\lstdefinelanguage{php}{
  keywords={typeof, new, true, false, catch, private, public, static, function, return, null, try, switch, var, if, for, foreach, in, while, do, else, elseif, case, break, self},
  ndkeywords={class, export, boolean, throw, implements, import, this},
  identifierstyle=\color{black},
  sensitive=false,
  comment=[l]{//},
  morecomment=[s]{/*}{*/},
  morestring=[b]',
  morestring=[b]"
}

\lstset{basicstyle=\scriptsize, lineskip={-2.0pt}, numbers=left, numberstyle=\tiny\color{mygray}, numbersep=5pt, captionpos=b,  basicstyle=\footnotesize\ttfamily,
  keywordstyle=\color{blue}\bfseries, 
  ndkeywordstyle=\color{darkgray}\bfseries,
  commentstyle=\itshape\color{green!40!black},
  stringstyle=\color{srcblue},breaklines=true}



\renewcommand{\lstlistingname}{Quellcode}% Listing -> Algorithm
\renewcommand{\lstlistlistingname}{\lstlistingname verzeichnis}% List of Listings

%% Diese Zeile unbedingt stehen lassen und anpassen - sie enth�lt Autor und Titel der Ausarbeitung
\mywork{David Bujok}{Erweiterung eines E-Learning-Moduls zum Assessment von
B-Baum-Datenstrukturen}

\begin{document}
\SetBgContents{}
	%% Bei Diplomarbeiten folgende Zeile nutzen
	\mybachelortitle{Dipl.-Wirt.Inform. Claus Alexander Usener}
	%% Bei Bachelorarbeiten diese Zeile auskommentieren
	%%\mybachelortitle{(ggf. Name des Betreuers)}
	%% Bei Seminararbeiten diese Zeile auskommentieren
	%%\myseminartitle{Titel des Seminars}{(ggf. Name des Betreuers)}

  %% Inhaltsverzeichnis
  %% frontmatter setzt die Seitenzahlen auf i, ii, ...
  \frontmatter

	%% Generiert automatisch aus den Sektionsbefehlen ein Inhaltsverzeichnis  
  \tableofcontents
  \newpage
%	\lstlistoflistings
%	\newpage
%	\listoffigures
%   \newpage

	%% das Mainmatter sorgt f�r die Nutzung arabischer Seitenzahlen
	\mainmatter
	
\section{Einleitung}
\label{sec:einleitung}

\glqq 
E-Assessment ist eines der Schlagworte der letzten Jahre\grqq \ \cite[S. 11]{Ruedel2010}.
Unter dieser Form des Assessments versteht man das Nutzen von Hardware und Software im Pr�fungsprozess. 
So wurden im deutschsprachigem Raum f�r E-Assessment Synonyme wie Online-Pr�fungen oder computergest�tzte Pr�fungen verwendet \cite[S. 13]{Ruedel2010}.
Zwei Gr�nde sprechen f�r diese Assessmentform als Mittel des Leistungsnachweises im Pr�fungsalltag.
Durch Effizienzsteigerung kann knappen Raum- und Personalressourcen vorbeugen. 
Au�erdem k�nnen neue didaktische Konzepte wie bessere Visualisierung oder neue Aufgabenformate, die Computer erst erm�glichen, angeboten werden \cite[S. 11 ff.]{Ruedel2010}. 
Die meisten Systeme, die E-Assessments anbieten, stellen jedoch nur einfache Aufgabentypen wie Multiple-Choice-Aufgaben zur Verf�gung. Da insbesondere technische Disziplinen wie die Informatik auf Aufgabentypen zur Erhebung von kognitiven F�higkeiten angewiesen sind \cite[S. 24]{Kuchen2010}, 
%Die steigende Anzahl von Studierenden bei gleichbleibender Zahl der Lehrpersonen und Assistierenden stellt Universit�ten vor neue Herausforderungen. Insbesondere die Logistik der steigenden Anzahl von Modulabschl�ssen bei knappen Raum- und Personalressourcen wird immer mehr zum Problem. Hierf�r bietet E-Assessment mit der M�glichkeit der Verbesserung und Erh�hung der Effizienz im Pr�fungsalltag einen m�glichen Ausweg. Unter E-Assessment versteht man das Nutzen von Hardware und Software im Pr�fungsprozess. So wurden im deutschsprachigem Raum f�r E-Assessment Synonyme wie Online-Pr�fungen oder Computergest�tzte Pr�fungen verwendet \cite[S. 11 ff.]{Ruedel2010}.
entwickelte der Lehrstuhl f�r Praktische Informatik in der Wirtschaft der WWU M�nster das E-Assessment-Tool EASy-DSBuilder zum Assessment von AVL-Baum-Datenstrukturen f�r die Moodle-Distribution  Learnweb \cite[S. 1]{Usener2014}. Bei Moodle handelt es sich um ein weltweit stark verbreitetes Lernmanagement-System \cite[S. 33]{Gertrsch2007}, das Lehrenden, Administratoren und Lernenden eine robuste, sichere und integrierte Plattform bereitstellen soll \cite{moo15d}.
%Insbesondere viele bekannte Lernplattformen unterst�tzen den �bungsbetrieb durch Funktionen zu Bereitstellung von Aufgabenbl�ttern und zur Organisation studentischer L�sungen. Einige Systeme bieten dar�ber hinaus eine Elektronische �berpr�fung der L�sung an. \cite{Kuchen2010}.
%Unter Lernplattformen ist Moodle international die am weitesten verbreitete Lernplattform \cite{hei15}. 
%Sie ist weltweit in 231 L�ndern �ber 53.000 Seiten registriert \cite{moo15a}
%Bei Moodle handelt es sich um ein Softwarepaket, welches einen  konstruktivistischen Lehr- und Lernansatz unterst�tzt. \cite{moo15a} 
%Weiterhin ist Moodle eine frei verf�gbare Open Source Software. %Dies bedeutet, dass sie frei erh�ltlich ist. 
Das System bietet die M�glichkeit der individuellen Anpassung an spezifische Anwendungssituationen \cite{moo15a}.
%Auch die Westf�lische Wilhelms-Universit�t M�nster stellt zur Verbesserung des Lehrbetriebs eine Moodledistribution unter dem Namen Learnweb zur Verf�gung.
%F�r die Vorlesung \textit{Informatik I} wurde bereits ein Moodlemodul implementiert, welches die M�glichkeit  bietet ????

Das Ziel dieser Arbeit ist die Entwicklung und Vorstellung eines E-Learning-Moduls zum Assessment von B-Baum-Strukturen.  
Im Zentrum steht hierbei das Moodle-Modul EASy-DSBuilder. Insbesondere stellt diese Arbeit die Erweiterung dieses Moduls um des Assessments von B-Baum-Datenstrukturen vor.
Grundlagen dieser Arbeit (Kapitel \ref{sec:grundlagen}) sind eine Einf�hrung in die Lernplattform Moodle, die Formen des E-Assessments, und schlie�lich die Definition einer B-Baum-Datenstruktur.
Das darauf folgende Kapitel \ref{sec:dsbuilder} stellt das Moodle-Modul EASy-DSBuilder vor. Hierbei wird auf die Funktionalit�t aus Benutzersicht und auf die Struktur aus technischer Sicht eingegangen. Im Kapitel \ref{sec:anforderungen} werden die Anforderungen an ein Tool zum Assessment von B-Baum-Datenstrukturen vorgestellt. Insbesondere werden im Kontext des EASy-DSBuilder �nderungsanforderung an dieses Modul zur Umsetzung des Assessments von B-Baum-Datenstrukturen dargelegt. Anschie�end wird die Umsetzung der Anforderungen betrachtet. Hierbei werden die entwickelten Funktionalit�ten beleuchtet und es wird ein Einblick auf ausgew�hlte Implementierungsdetails gegeben. Abschie�end wir ein Fazit des Ergebnisses dieser Arbeit betrachtet und ein Ausblick auf m�gliche weitere Entwicklungen des Moduls gegeben.


\section{Grundlagen}
\label{sec:grundlagen}
Im folgenden Kapitel werden f�r dies Arbeit wichtige Grundlagen vorgestellt. Insbesondere handelt es sich um  Einf�hrungen in die Lernplattform Moodle, das E-Assessment und die B-Baum-Datenstruktur.


\subsection{Die Lernplattform Moodle}
\label{sec:moodle}
Der folgende Abschnitt erl�utert, um welche Art System es sich bei der Lernplattform Moodle handelt und stellt weitere wichtige Eigenschaften dieser Plattform dar.

\subsubsection{Was ist Moodle}
Bei Moodle handelt es sich um ein weltweit anerkanntes Lernmanagement-System \cite[S. 33]{Gertrsch2007}, das Lehrenden, Administratoren und Lernenden eine robuste, sichere und integrierte Plattform bereitstellen soll \cite{moo15d}. Der Name Moodle leitet sich von der Akronymisierung des Ausdrucks \textit{\textbf{M}odular \textbf{O}bject \textbf{O}riented \textbf{D}ynamic \textbf{L}earning \textbf{E}nvironment} ab \cite[S. 33]{Gertrsch2007}.
Moodle ist weiterhin eine frei verf�gbares Softwarepaket, da es der GNU Public Lizenz unterliegt \cite{Scheb2009}. Software, welche unter einer GNU Public License vertrieben wird, darf kopiert, benutzt und weiterentwickelt werden. Eine einschr�nkende Bedingung ist, dass �nderungen oder Weiterentwicklungen den eben genannten Pflichten unterliegen, sie folglich auch ver�ffentlicht und Dritten zur Verf�gung gestellt werden m�ssen \cite{moo15a}. Die Plattform wird von einer weltweiten Gemeinschaft und von der Moodle Pty. Ltd. laufend weiterentwickelt. Vom australischen Moodle Erfinder Marign Dougiamas wird das Projekt zielgerichtet geleitet. Des weiteren gibt es ein Netzwerk professioneller Partnerunternehmen, welche Support und Beratung leisten \cite[S. 12]{Scheb2009}.

\subsubsection{Moodle als Lernmanagement-System}
Unter einem Lernmanagement-System (LMS) versteht man im wesentlichen ein Management-System f�r die Automatisierung und die Administration von Ausbildung. Insbesondere sollten LMS �ber folgende Funktionen verf�gen \cite[S. 14]{Schulmeister2005}:

\begin{itemize}
\item Eine Benutzerverwaltung (Anmeldung mit Verschl�sselung)
\item Eine Kursverwaltung (Kurse, Verwaltung der Inhalte, Dateiverwaltung)
\item Eine Rollen- und Rechtevergabe mit differenzierten Rechten
\item Kommunikationsmethoden (Chat, Foren) und Werkzeuge f�r das Lernen (Whiteboard, Notizbuch, Annotationen, Kalender etc.)
\end{itemize}
Moodle stellt diese Funktionen zur Verf�gung. So besteht �ber die Website-Administration die M�glichkeit der Benutzerverwaltung \cite[S. 563
 2 ff.]{Gertrsch2007} und der Kursverwaltung \cite[S. 588 ff.]{Gertrsch2007}. Bei der Rollen - und Rechtevergabe bietet Moodle flexible M�glichkeiten der Administration. So verf�gt Moodle �ber vorgefertigte Basisrollen mit bestimmten Rechten, die einen Gro�teil der Anwendungsf�lle abdecken. F�r bestimmte Situationen k�nnen Rollen jedoch editiert oder neue Rollen erstellt werden \cite[S. 191]{Gertrsch2007}. Die Basisrollen des Systems sind \cite[S. 193]{Gertrsch2007}:
\begin{itemize}
\item \textit{Kursverwalter:} Wer in einem Kontext \textit{Kursverwalter} ist, kann einen \textit{neuen Kurs erstellen} und in diesem unterrichten, weil er automatisch als \textit{Trainer} eingetragen wird. Zu anderen Kursen im gleichen Kontext hat er aber keinen Zugriff.
\item \textit{Trainer}: Wer in einem Kontext \textit{Trainer} ist, ist in s�mtlichen Kursen dieses Kontextes als \textit{Trainer} eingetragen und kann diese Bearbeiten
\item \textit{Trainer ohne Editorrecht}: ist Trainer in s�mtlichen Kursen dieses Kontextes. 
\item \textit{Teilnehmer/in}: ist Teilnehmer in s�mtlichen Kursen dieses Kontextes, kann also auch Kurse mit Zugriffsschl�ssel betreten.
\end{itemize}
Auf die Kommunikationsmethoden, die Moodle zur Verf�gung stellt, wird in Kapitel \ref{sec:kommunikationsmethoden} eingegangen. 
 

Abbildung \ref{fig:architekurLMS} zeigt die idealtypische Architektur eines LMS. Zu sehen ist, dass ein LMS �ber drei Schichten verf�gt. Bei der untersten Schicht handelt es sich um die Datenbankschicht, in der alle Lernobjekte, Benutzerdaten und andere gehalten werden. Die mittlere Schicht stellt Schnittstellen zur Verf�gung. Die oberste Schicht stellt die Sicht bereit, �ber die �ber die seitens von Administratoren, Dozenten oder Studierenden auf Inhalte zugegriffen werden kann \cite[S. 11]{Schulmeister2005}.
\begin{figure}[htbp] 
  \centering
     \includegraphics[width=0.9\textwidth]{graphics/ArchitekturLMS.jpg}
  \caption{Idealtypische Architektur eines LMS \cite[S. 12]{Schulmeister2005}}
  \label{fig:architekurLMS}
\end{figure} 
Im Kontext dieser Arbeit wird insbesondere verst�rkt auf die Schnittstellenschicht eingegangen. Das Kapitel \ref{sec:modularitaet} wird die M�glichkeit Einbindung von Modulen in Moodle erl�utern. Das Kapitel \ref{} wird hingegen den Teilbereichen Aufgeben und Tests aus dem Bereich Authoring der Ansichtsschicht auseinandersetzen. \textit{Es wird der Forschungsbereich E-Assessment vorgestellt, welcher sich mit �berpr�fungen �ber Onlinemedien auseinandersetzt}.




\subsubsection{Kommunikationsmethoden in Moodle}
\label{sec:kommunikationsmethoden}


\subsubsection{Modularten}
\label{sec:modularten}
Neben den eingangs beschriebenen Activity modules gibt es noch weitere Plugin-
arten in Moodle, von denen eine Auswahl nun n�her beschrieben wird.

\paragraph{Activity Plugin} \hfill \\
\textit{Activity Plugins} bieten in erster Linie Kursaktivit�ten wie Foren, Hausarbeiten, Quizze und Workshops \cite{moo15g}.  


\paragraph{Bl�cke} \hfill \\
\textit{Bl�cke} sind Plugins, welche eine einfache Art Interface besitzen und auf jeder
Moodle-Seite hinzugef�gt werden k�nnen. Zur Platzierung von Bl�cken werden zu-
meist zwei Spalten auf einer solchen Moodle-Seite angeboten, jeweils eine links und
eine rechts vom eigentlichen Inhalt der Seite. Diese Pluginart dient dabei h�ufig
der Darstellung von zus�tzlichen Inhalten oder bietet eine weitere Sicht auf schon
vorhandene Daten, welche andere R�ckschl�sse zul�sst. Mithilfe von Bl�cken ist es
zudem m�glich, ein Interface zur Bearbeitung von Daten anzubieten \cite{Moo14f}.

\paragraph{Themes} \hfill \\
Sogenannte \textit{themes} bilden eine weitere Pluginart, mithilfe derer sich die Darstellung und das Aussehen von Moodle auf verschiedenen Ebenen anpassen l�sst. Bei
diesen Ebenen, auf denen ein \textit{theme} angewendet wird, kann es sich sowohl um einzelne Moodle- oder Kursseiten als auch um alle Kursseiten einer Kurskategorie handeln.
Durch diese Art von Plugin schafft Moodle die Trennung der Pr�sentationsschicht
von der funktionalen Schicht \cite{Moo14f}.

\paragraph{Course formats} \hfill \\
Mithilfe von \textit{course format} Plugins l�sst sich die Struktur von Kursen anpassen.
Diese Plugins legen fest, wie die Hauptseite eines Kurses aufgebaut ist, wie der Navigationsbaum innerhalb eines Kurses erstellt wird und wie die einzelnen Kursbereiche
strukturiert sind \cite{Moo14b}.

\paragraph{General plugins} \hfill \\
Eine weitere wichtige Pluginart, sind die sogenanten \textit{general plugins}, oder auch \textit{local Plugins}. Diese Art
von Plugin kommt immer dann zum Einsatz, wenn keine der sonstigen Pluginarten
passt. Dies kann unter anderem dann der Fall sein, wenn der Navigationsblock um
ein eigenes Men� erweitert werden soll. Auch f�r Webservices und die Einbindung
externer Systeme sind \textit{local-Plugins} die erste Wahl \cite{Moo14e}.

\subsubsection{Aufbau eines Moduls}
\label{sec:aufbauPlugin}

F�r jedes Plugin in Moodle muss eine bestimmte Datenstruktur implementiert werden. Diese besteht aus separaten Unterverzeichnissen und verpflichtenden Dateien. Des weiteren haben Entwickler die M�glichkeit weitere Dateien selbst zu gestalten \cite{moo15b}. 
\hfill \\ \hfill \\ 
\pfile{/<modname>/backup} \hfill \\ 
Dieser Ordner dient zur Ablage aller Dateien, welche definieren, wie sich das Modul bei einem Backup oder einer Wiederherstellung verhalten soll \cite{moo15b}.

\hfill \\ 
\pfile{/<modname>/db}
\begin{itemize} 
	\item \pfile{/access.php} In dieser Datei werden die so genannten \textit{capabilities} f�r das Plugin definiert. \textit{capabilities} beschreiben die Berechtigungen, welche eine Rolle in diesem Plugin zugeordnet bekommt. Eine Berechtigung ist beispielsweise das hinzuf�gen einer neuen Instanz diese Plugins zu einem Kurs \cite{moo15b}.
	\item \pfile{/install.xml} Diese Datei wird bei der Installation des Moduls benutzt. Sie definiert, welche Datenbanktabellen und -felder erstellt werden. Hierf�r wird das XML-Format verwendet. Braucht das Modul keine weiteren Tabellen oder Spalten, so kann auf diese Datei verzichtet werden \cite{moo15b}.
	\item \pfile{upgrade.php} Auf Grund dessen, dass die Datei \pfile{install.xml} nur einmal w�hrend der Installation aufgef�hrt wird, braucht es eine Methode um die Datenbank nachtr�glich um Tabellen oder Spalten zu erweitern. Diese Funktionalit�t wird von dieser Datei bereitgestellt und kommt bei einem Update des Moduls zum Einsatz \cite{moo15b}.
\end{itemize}
\hfill \\ 
\pfile{/<modname>/lang} \hfill \\ 
In diesem Ordner k�nnen alle \textit{Strings} gespeichert werden, die im Modul benutzt werden sollen. Jede Sprache hat hierbei einen spezifischen Ordnernamen ('\pfile{/lang/de}' beispielsweise f�r die Sprache Deutsch). Die in diesem Ordner gespeicherte Datei muss in der Form \pfile{<modname>.php} benannt sein \cite{moo15b}.

\hfill \\ 
\pfile{/<modname>/pix} \hfill \\ 
Dieser Ordner dient dazu das Logo des Moduls zu speichern, welches neben dem Modulname erscheint. Der Name des Logos muss \pfile{icon.gif} lauten. Weiterhin besteht die M�glichkeit weitere Bilder in diesem Ordner zu speichern
\cite{moo15b}.

\hfill \\ \hfill \\ 
\pfile{/<modname>}
\begin{itemize} 
	\item \pfile{/lib.php} Diese Datei bietet eine Schnittstelle f�r die zu implementierenden Kernfunktionen. Kernfunktionen werden dazu ben�tigt, damit das Modul in Moodle integriert arbeiten kann. Diese
Schnittstellen-Funktionen werden von Moodle nach einem bestimmten Ereignis im Prozessablauf aufgerufen, sofern diese vom Modul in der Datei \pfile{/lib.php} definiert wurden. Dabei ist jeder dieser Funktionen zun�chst der Name des
Moduls vorangestellt, gefolgt von einem Unterstrich und dem Funktionsnamen
(\pfile{<pluginname>\_core\_function}). Diese Konvention ist deshalb so wichtig, da
die Datei \pfile{/lib.php} keine Klasse definiert, welche Namenskonflikte verhindern. Es wird geraten Funktionalit�ten, welche einen hohen Codeumfang haben, der �bersichtlichkeit halber in eine Datei namens \pfile{locallib.php} auszulagern.
w�rde. \cite{moo15c}
	
	\item \pfile{/mod\_form.php}  Diese Datei wird beim Hinzuf�gen oder Bearbeiten eines Kurses genutzt. Es enth�lt die Elemente welche im Editiermen� des Moduls zu sehen sind. Die in dieser Datei enthaltende Klasse muss der Namenskonvention nach in der Form 
	\pfile{mod\_<modname>\_mod\_form} benannt sein.
	
	\item \pfile{/index.php} Diese Datei wird von Moodle dazu genutzt, um auf Aktivit�ten bei allen Instanzen dieses Moduls, welche einem bestimmten Kurs �bergeben wurden, zu h�ren. Diese Datei ist spezifisch f�r diese Modulart \textit{Activity Module}.
	
	\item \pfile{/view.php} Diese Datei wird bei der Erzeugung der Anzeige ben�tigt. Beim Aufrufen eines Moduls �ber die Kurssicht wird auf diese Datei verwiesen.  Dabei wird dieser Datei die Instanz-ID �bergeben, anhand welcher
die Daten der Instanz ausgew�hlt und angezeigt werden k�nnen. Diese Datei ist spezifisch f�r diese Modulart \textit{Activity Module}.
	
	\item \pfile{/version.php} Diese Datei enth�lt die aktuelle Versionsnummer dieses Moduls. Au�erdem enth�lt diese Datei weitere Attribute wie beispielsweise die Mindestanforderungen hinsichtlich der Moodleplattform.
	
	
\end{itemize}

\subsection{E-Assessment}
\label{sec:eassessement}


\subsubsection{Einleitung}
Auf Grund des Bologna-Prozesses, Sparma�nahmen, Diskussionen �ber die Verwendung von Studiengeb�hren und steigende Studierendenzahlen wird im Bereich der �bungsangebote der Einsatz von E-Assessments vermehrt in Betracht gezogen \cite[S. 9]{kortemeyer2010}. 
Unter e-Assessment versteht man das Spektrum der auf den neuen (elektronischen) IKT basierenden Verfahren der lehrzielbezogenen Bestimmung, Beurteilung, Bewertung, Dokumentation und R�ckmeldung der jeweiligen Lernvoraussetzungen, des aktuellen Lernstandes oder der erreichten Lernergebnisse/ -leistungen vor, w�hrend (?Assessment f�r das Lernen?) oder nach Abschluss (?Assessment des Lernens?) einer spezifischen Lehr-Lernperiode \cite[S. 6]{Bloh2006}. E-Assessment-Systeme k�nnen des Weiteren Nutzen f�r Lehrende und Lernende bieten. Cook und Jenkins identifizierten neuen Vorteile. Bei den Wichtigsten handelt es sich um die M�glichkeiten des direkten Feedbacks und der sofortigen und objektiven Benotung.  Au�erdem bieten E-Assessment-Systeme einfache Skalierbarkeit und Wiederverwendbarkeit \cite[S. 3]{Cook2010}. E-Assessments k�nnen hinsichtlich ihrer Hauptaufgabe in drei Typen unterteilt werden \cite[S. 8]{Cook2010}:
\begin{itemize}
\item \textbf{Diagnostische Assessments} finden normalerweise zu beginn einer Lehrveranstaltung statt um m�gliche Wissensl�cken bei Teilnehmern aufzudecken und gegebenenfalls ein Nachbesserungsangebot zu schaffen. 
\item \textbf{Formative Assessments} erm�glicht Studierenden und Lehrenden einen �berblick �ber den aktuellen Lernstand zu erhalten. E-Assessment erm�glicht Studierenden weiterhin ein direktes Feedback.
\item \textbf{Summative Assessments} bieten eine Bewertungsgrundlage �ber den Lernfortschritt eines Studierenden. Bei diesem Typen steht im Gegensatz des Feedbacks die Notengebung im Vordergrund.
\end{itemize}
\textbf{So wird in den folgenden Abschnitten auf formative Assessment eingegangen, indem E-Assessments als Teilbereich des Lernprozesses vorgestellt werden, und anschlie�end summative Assessments behandelt werden, indem tiefgehender auf den Aspekt des Leistungsnachweises im Zusammenhang mit E-Assessment eingegangen wird.}

\subsubsection{E-Assessment als Teilbereich des Lernprozesses}
In der Hochschullehre nehmen Leistungs�berpr�fungen einen integralen Bestandteil in Lehr- und Lernprozessen ein. Ein Augenmerk liegt hierbei auf der Identifizierung und Bewertung individueller Lernfortschritte \cite[S. 23 f.]{Kuchen2010}. Darunter wird insbesondere das Pr�fen und Bewerten einzelner �bungen als Teil einer p�dagogischen Handlungseinheit verstanden. Dabei k�nnen die verschiedenen �bungsphasen und -iterationen unterschiedliche L�ngen haben und inhaltliche Abh�ngigkeiten haben \cite[S. 25]{Kuchen2010}. Abbildung \ref{fig:Uebungsbetrieb} zeigt den Lehr-Lern-Prozess eines �bungsbetriebs.

\begin{figure}[htbp] 
  \centering
     \includegraphics[width=0.9\textwidth]{graphics/Uebungsbetrieb.jpg}
  \caption{Iterative Lehr-Lern-Prozesse eines �bungsbetriebs}
  \label{fig:Uebungsbetrieb}
\end{figure} 

Im Regelfall beginnt ein �bungszyklus mit der Phase der Lehre. In dieser Phase vermittelt der Lehrende dem Studierenden die entsprechenden Inhalte. Anschlie�end hat der Studierende in der Lernphase die M�glichkeit das vermittelte Wissen im Selbststudium zu vertiefen. In einer darauf folgenden �bung kann der Studierende sein theoretisches Wissen anhand geeigneter praktischer �bungen ausprobieren und festigen. Im Anschluss folgt die Diagnose der �bung, welche Korrektur und Benotung beinhaltet. Dies geschieht im Regelfall durch DozentInnen oder TutorInnen, kann aber auch durch KommilitonInnen erfolgen (Peer Review). Es sollte ein Feedback �ber die erbrachte Leistung und Ratschl�ge folgen. Optional kann aus den Ratschl�gen eine Leistungsvereinbarung abgeleitet werden \cite[S. 25 f.]{Kuchen2010}.



Weiterhin sind Leistungs�berpr�fungen ein integraler Bestandteil der Lehr- und Lernprozesse. So sollen w�chentliche �bungszettel dazu beitragen, dass das von Studierenden erlernte theoretische Wissen durch Bearbeitung geeigneter Aufgaben reflektiert und verinnerlicht wird. Insbesondere gestaltet sich das Halten von klassischen Pr�senz�bungen bei knappen finanziellen Ressourcen schwer. Eine M�glichkeit trotz Reduktion des Aufwands eine langfristige hohe Qualit�t des �bungsbetriebs zu gew�hrleisten ist in E-Assessment zu sehen. \cite[S. 24]{Kuchen2010}. 
In den meisten Systemen, welche E-Assessment anbieten werden jedoch nur einfache Aufgabentypen wie Multiple-Choice-Aufgaben oder L�ckentexte unterst�tzt. �ber solche Aufgabentypen k�nnen Wissenszuw�chse gut gepr�ft werden, kognitive F�higkeiten und Methodenwissen jedoch nur bedingt. Insbesondere in naturwissenschaftlichen oder technischen Disziplinen wie der Informatik ist die Kontrolle und �berpr�fung von Aufgaben, bei denen die Anwendung analytischer, kreativer und konstruktiver F�higkeiten gefordert ist besonders wichtig. So entwickelte beispielsweise die WWU M�nster das E-Assessment-System EASy f�r das Informatikstudium, welches durch die Bereitstellung von einfachen Aufgabentypen wie dem Multiple-Choice, aber auch anspruchsvoller Aufgabentypen zu mathematischen Beweisen und Programmierung\cite[S. 24]{Kuchen2010}. So zeigte die Evaluation dieses Systems im Praxiseinsatz, dass der Lernprozess unterst�tzt wurde und insbesondere der �bungsbetrieb von diesem System profitiert hat. Weiterhin stellte die Evaluation in Ausblick, dass solche Systeme auch auf eine Vielzahl von anderen Bildungseinrichtungen �bertragen werden kann \cite[S. 34]{Kuchen2010}. 


\subsubsection{E-Assessment als Methode des Leistungsnachweises}
Neben Leistungs�berpr�fungen als Bestandteil des Lehr- und Lernprozesses kann E-Assessment auch f�r einen Leistungsnachweis herangezogen werden. Als Leistungsnachweis wird die abschlie�ende Leistungserbringung zum Bestehen einer Lehrveranstaltung oder eines Modul angesehen. So werden Leistungsnachweise klassisch durch Klausuren, Referate oder Studienarbeiten repr�sentiert. Wir also E-Assessment als Leistungsnachweis eingesetzt, kann man davon sprechen, dass anstatt einer traditionellen Pr�fungsform eine vergleichbare elektronische Durchf�hrungsvariante gew�hlt wurde \cite[S. 18]{Ruedel2010}. Der folgende Abschnitt besch�ftigt sich mit der Gegen�berstellung von traditionellen Pr�fungsformen und elektronischen �quivalenten. 

Ein klassisches Beispiel f�r einen Leistungsnachweis ist die schriftliche Pr�fung. Ihr elektronisches �quivalent ist die E-Pr�fung bzw. E-Klausur, welche offene und geschlossene Fragen verwendet um Lehrinhalte zu pr�fen. Oft wird sie als computerunterst�tzende Pr�fung mit Multiple-Choice-Fragen verstanden, jedoch sind auch offene Fragen sehr effektiv. So kann beispielsweise durch die verbesserte Lesbarkeit die Korrekturzeit verk�rzt werden. Weiterhin kann unter Elektronic Submission der automatisierte elektronische Abgabeprozess von schriftlichen Arbeiten wie Protokollen, Seminar- und Hausarbeiten verstanden werden. Hierbei kann beispielsweise in Lernplattformen Ablagem�glichkeiten zur fristgerechten Abgabe bereitgestellt werden. Daneben k�nnen  Simulationen am Computer komplexe wissenschaftspraktische T�tigkeiten �bernehmen. Derweise k�nnen Simulationen die Bewertung typischer Labort�tigkeiten wie beispielsweise die Handhabung von Mikroskopen oder die Interpretation von R�ntgenbildern �bernehmen. Au�erdem k�nne beispielsweise Foren als Bewertungsm�glichkeit von m�ndlicher Mitarbeit dienen. Ebenso kann k�nnen Wikis eine Pr�sentationsm�glichkeit einer erbrachten Gruppenleistung dienen. So kann abschlie�en E-Assessment als Methode der Leistungserbringung als Substitutionsmodell verschiedener klassischer Beurteilungsmethoden verstanden werden \cite[S. 18 ff.]{Ruedel2010}.



\subsection{Die Datenstruktur B-Baum} 
\label{sec:bbaum}
Der B-Baums ist eine von R. Bayer und E. McCreight entwickelte ausgeglichene Baumstruktur. Hierbei steht der Name \textit{B} f�r balanciert, breit, buschig ober Bayer, nicht jedoch f�r bin�r. Im Gegensatz zu Bin�rb�umen ist die Grundidee eines B-Baums der variierende Verzweigungsgrad, w�hrend die Baumh�he vollst�ndig ausgeglichen ist. Dies bedeutet, dass alle Pfade von der Wurzel zu den Bl�ttern gleich lang sind, Knoten jedoch mehrere Kanten enthalten k�nnen. Deswegen fallen B-B�ume auch in die Kategorie Mehrwegb�ume \cite[S. 386]{Saake2014}. 
\noindent \\ \\
Ein Baum T ist ein \textit{B}-\textit{Baum der Ordnung} $k, k \in \mathbb{N}, k\geqslant 2$, falls 
\begin{enumerate}
\item alle Bl�tter dieselbe Tiefe,
\item die Wurzel mindestens 2 und jeder andere innere Konten mindestens $k$ S�hne und
\item jeder innere Knoten h�chstens $2k -1$ S�hne

\end{enumerate}
haben \cite[S. 21]{Blum2013}.
\begin{figure}[htbp] 
  \centering
     \includegraphics[width=0.9\textwidth]{graphics/bbaum.png}
  \caption{Struktur eines B-Baums}
  \label{fig:bbaum}
\end{figure}
F�r einen sortierten Baum muss folgendes gelten: 
Sei \textit{u} ein innerer Knoten mit \textit{l} S�hnen. Bezeichne $T_{i}, 1 \leqslant i \leqslant l$ den $i$-ten Unterbaum des Teilbaumes mit Wurzel \textit{u}. Der Knoten \textit{u} enth�lt $l-1$ Schl�ssel $s_{1}, s_{2},..., s_{l-1}$ und $l$ Zeiger $z_{1}, z_{2},..., z_{l}$. Der Zeiger $z_{i}$ zeigt auf den $i$-ten Sohn von $u$. F�r alle Knoten $v$ in $T_{i}$ gilt f�r alle Schl�ssel $s$ in $v$ \cite[S. 22]{Blum2013}:

$\begin{cases}
s\leqslant s_{i} & \text{falls } i=1\\
s_{i-1}<s\leqslant s_{i}& \text{falls } 1<i<l\\
s_{l-1}<s &\text{falls } i=l
\end{cases}$

\section{Vorstellung des Moodleplugins EASy-DSBuilder}
\label{sec:dsbuilder}
Der EASy-DSBuilder ist ein E-Assessent Tool, welches der Evaluation grundlegender Konzepte �ber Operationen (z.B. Suchen, Einf�gen, und Entfernen) innerhalb der Datenstruktur \textit{Bin�rbaum} dient \cite{Usener2014}. 

Das Tool wurde speziell f�r die Lernplattform Moodle implementiert. 

Diese Kapitel wird das Tool EASy-DSBuilder vorstellen. Hierbei wird zu erst in Kapitel \ref{sec:funktionalitaet} auf die Funktionalit�t aus Benutzersicht eingegangen. Anschlie�end erfolgt eine Erl�uterung der Implementation (Kapitel %\ref{?}).

\subsection{Funktionalit�t aus Benutzersicht}
\label{sec:funktionalitaet}

Im folgenden Kapitel wird die Funktionalit�t des Moodleplugins EASy-DSBuilder vorgestellt. Hierbei wird auf die beiden Sichten Student und Lehrender eingegangen.

\paragraph{Lehrender} \hfill \\
Der Lehrende hat zwei Grundlegend Aufgaben. Zum einen ist er daf�r verantwortlich, dass eine Aufgabe erstellt wird, zum anderen hat er die M�glichkeit, die Ergebnisse einzusehen, um beispielsweise Indikatoren zur Verbesserung der Lehre zu finden \cite{Usener2014}. Wird eine neue Aufgabe erstellt, hat der Lehrende die M�glichkeit allgemeine Informationen wie den \textit{Titel}, die \textit{Beschreibung} und das \textit{F�lligkeitsdatum} anzugeben. Unter \textit{Source Files} kann der Lehrende �ber Drag-and-Drop seine eigene Implementierung einer Datenstruktur zu dem Moodleplugin hinzuf�gen. Hierzu muss er jedoch eine Wrapper auf Basis eines Interfaces implementieren, welches die verlangten Voraussetzungen erf�llt. Diese Wrapperklasse muss anschlie�end vom Lehrenden als Hauptklasse eingestellt werden. Auf die Funktionalit�t der Wrapperklasse aus technischer Sicht wird im Kapitel \ref{sec:technWebService} n�her eingegangen. Des weiteren kann der Lehrende eine Feedback aktivieren. Die genau Funktionalit�t des Feedbacks wird im Absatz der Studentensicht erl�utert.  

\paragraph{Studierender} \hfill \\
Der Studierende verf�gt �ber zwei Ansichten. Zum einen die �bersichtsansicht, zum anderen die Bearbeitungsansicht.
Nachdem der Studierende sich in das Plugin eingew�hlt hat, ist ist �bersichtsansicht �ber den bisherigen Verlauf des Assessments zu sehen. In dieser �bersicht ist der Abgabestatus, der Bewertungsstand, der Abgabezeitpunkt und die verbliebene Zeit zu sehen (vergl. Abb. \ref{}). �ber den Button \textit{Aufgabe bearbeiten} gelangt der Studierende zum Editor, in dem die Aufgabe bearbeitet werden kann. 

Die Bearbeitungsansicht ist in drei grundlegende Abschnitte unterteilt. Den oberen Teil der Ansicht bildet ein �berblick �ber den aktuellen Schritt. Dieser �berblick beinhaltet den Fortschritt der Aufgabe, die Nummer des aktuellen Schritts und den aktuellen Arbeitsauftrag. Im mittleren Teil der Sicht befindet sich der Editor, in dem der Studierende die Aufgabe bearbeiten kann. Im oberen linken Bereich des Editor befinden sich drei Kn�pfe (vergl. Abb. \ref{}.1), �ber welche der Editiermodus ausgew�hlt werden kann. Der 1. Knopf erm�glicht das verschieben von Konten im Editor, der zweite Knopf erm�glicht das Ziehen von Kanten zwischen zwei Knoten, und der dritte Knopf erm�glicht das Entfernen von Konten.

Der DragandDropGrafikeditor enth�lt zwei bearbeitbare Elemente, die Knoten und die Kanten. �ber Manipulation dieser Elemente sollen Studierende den
Umgang mit Datenstrukturoperationen erlernen. Hierbei kann der Studierende Operationen wie das Einf�gen in oder das L�schen aus einer Datenstruktur 
praktizieren. In der \textbf{momentanen} Version des EASyDSBuilders beginnt jeder Schritt mit dem Ergebnisbaum des zuvor eingereichten Schrittes oder einem Initiierngsbaum 
wenn, es sich um den  ersten Schritt handelt.
Auf der linken Seite des Editors wird der einzuf�gende Knoten bereitgestellt. Die Aufgabe des Studierenden ist es, diesen Knoten an der richtigen 
Stelle in den Baum einzuf�gen. \textbf{Erl�uterung der M�glichkeiten von Manipulationen}
Nachdem der Studierende seine Ver�nderungen vorgenommen hat, kann er �ber den Knopf \textit{Syntax pr�fen} den Baum ausbalanciert anzeigen lassen. Auf diese Weise kann der Studierende �berpr�fen, ob die Anwendung den Baum im Sinne des Studierenden verarbeitet hat. Entspricht die �berpr�fte Struktur nicht der Struktur eines Baumes, \textbf{genauere Definiton} bekommt der Studierende eine Fehlermeldung mit Hinweis �ber die Fehlerquelle.

Hat der Lehrende bei der Einrichtung des DSBuilders die Option \textit{direktes Feedback} eingestellt, erscheint im Falle einer falschen Eingabe ein Feedbackfeld unterhalb des Editors. In diesem Feedbackfeld wird zu erst ein Informationstext angezeigt, welches das richtige Vorgehen in dem zuvor eingereichtem Schritt beschreibt. Unterhalb dieses Informationstextes ist der korrekte Baum zu sehen. Die falsch eingeordneten Knoten sind rot markiert. 


\subsection{Umsetzung aus technischer Sicht}
\label{sec:technologien}

Das gesamte System um den EASy-DSBuilder besteht backendseitig aus zwei separaten Systemen. Zum einen gibt es das eigentliche Moodleplugin, welches in eine bestehenden Moodelplattform integriert werden kann, zum anderen gibt es einen Datenstruktur-Verarbeitungsservice, welcher als Webservice implementiert ist. Das Moodleplugin hat die M�glichkeit �ber die Moodle-API Daten in einer SQL-Datenbank - beispielsweise einem MySQL-Server - zu hinterlegen. Die Kommunikation zwischen dem Moodleplugin und dem Webservice l�uft �ber das SOAP-Protokoll. Der Webservice ist als WildFly Application Server implementiert und unterliegt somit dem Java-EE7-Standard \cite{wildflyCert}. In Abbildung \ref{fig:technoloieUeberblick} ist dargestellt, wie die unterschiedlichen Technologien in einander greifen.

\begin{figure}[htbp] 
  \centering
     \includegraphics[width=0.9\textwidth]{grahics/UeberblickTechnologien.jpg}
  \caption{Technischer �berblick}
  \label{fig:technoloieUeberblick}
\end{figure}

Der Datenstruktur-Verarbeitungsservice hat die Aufgabe Datenstrukturen mit Hilfe von Die Separierung des Systems erfolgt aus den Risiken, dass der Code sch�dlich sein oder eine schlechte Ausf�hrungsleistung aufweist kann. Durch die Trennung der beiden Systeme kann in beiden F�llen Zusammen- oder Performanceeinbr�chen der gesamten E-Learning-Plattform vorgebeugt werden. Weiterhin kann so Datendiebstahl vorgebeugt werden, da in der Verarbeitungsumgebung keine nutzerbezogenen Daten verarbeitet werden. Bei Ausfall des Verarbeitungsservices ist jedoch das Aufrufen eines n�chsten Schrittes nicht mehr m�glich \cite{Usener2014}.

Auf Clientseite wird HTML mit CSS und JavaScript verwendet, um das Plugin f�r den Benutzer darstellen zu k�nnen. Als JavaScript-Frameworks wird jQuery und und als JavaSrcipt-Applikation wird jsdot eingesetzt. �ber jQuery ist die Kommunikation mit dem Moodleplugin �ber das AJAX-Protokoll organisiert. Jsdot dient als Grapheditor.  

\subsubsection{Datenstruktur-Verarbeitungsservice}  
\label{sec:technWebService}

Der Datenstruktur-Verarbeitungsservice kompiliert und f�hrt den vom Lehrenden bereitgestellten Code aus. Er ist als Webservice implementiert und kann somit von einem anderen Server aus bereitgestellt werden. Die Ausf�hrung des Codes ist vor jedem Einf�gen oder L�schen, das von einem Studierenden durchgef�hrt wird, notwendig.

Auf der Grundlage des bisherigen, eingereichten Schritts berechnet die Ausf�hrungsumgebung den n�chsten  die Ausf�hrungsumgebung der n�chsten Betriebswert (Taste, die eingef�gt oder gel�scht wird), die erwartete L�sung und die entsprechende detaillierte R�ckmeldungen.

\subsubsection{Moodleplugin backendseitig}
\label{sec:dsbuilderbackend}
Das backendseitige Moodleplugin besitzt die grundlegende Struktur eines Moodleplugins, wie sie in Kapitel \ref{sec:aufbauPlugin} dargestellt wurde. 

�ber die Datei \textbf{view.php} wird der EASy-DSBuilder initialisiert. Von hier aus wir die Datei \textbf{renderer.php} angesto�en. Diese Datei sorgt daf�r, dass die Ansicht erstellt wird. 

\textbf{AJAX}
Zur Kommunikation stellt das Moodleplugin eine AJAX Api zur Verf�gung. 

Der Codeausschnitt \ref{code:ajax} zeigt die Aktionen, die nach einer AJAX-Anfrage durchgef�hrt werden k�nnen. Der erste �berpr�ft, ob es sich bei der �bergebenen Struktur um eine richtige handelt

\lstinputlisting[language=PHP, caption=AJAX API, frame=single, label=code:ajax]{code/ajax.php}


%\end{scriptsize}



\subsubsection{Moodleplugin frontendseitig}
\label{sec:dsbuilderfrontend}


\section{�nderungsanforderungen des DSBuilders}
\label{sec:anforderungen}

In diesem Kapitel werden die Anforderungen an die Erweiterungen des EASy-DSBuilders vorgestellt und n�her erl�utert. Die Hauptanforderung lautet:
\begin{quote}
Der EASy DSBuilder soll um die Datenstruktur B-Baum erweitert werden.
\end{quote}
Aus dieser Hauptanforderung lassen sich weitere Unteranforderungen Ableiten.

\begin{itemize}
\item editirbare graphische Oberfl�che zur Erstellung von B-B�umen
\item Funktionalit�t soll beibehalten werden:
\begin{itemize}
\item Kommunikation Moodle <---> Web-Server
\item Schritte werden gespeichert
\item Eingabe �berpr�fen
\item Feedback
\item Schritt zur�ck
\end{itemize}
\item Lehrender: Auswahl zwischen Typ
\end{itemize}
\section{Umsetzung der �nderungsanforderungen}
Dieses Kapitel stellt die Stellen vor, an denen �nderungen vorgenommen werden mussten und erl�utert die Ursachen, auf Grund derer die �nderungen vorgenommen werden mussten. Hierzu wird zu erst in Abschnitt \ref{sec:umsetzungUeberblick} ein �berblick �ber die notwendigen �nderungen und Erweiterungen gegeben. In den restlichen Abschnitten wird tiefgehender auf die wichtigen �nderungen und Erweiterungen eingegangen. Dabei wird zuerst die Funktionalit�t erl�utert. Zur Verdeutlichung werden anschlie�end exemplarisch Ausschnitte aus der Implementierung vorgestellt.




\subsection{�berblick �ber �nderungen}
\label{sec:umsetzungUeberblick}
In diesem Abschnitt wird ein grundlegender �berblick �ber alle �nderungen gegeben, die in dem Moodlemodul EASy-DSBuilder vorgenommen wurden, um die Funktionalit�t des Assessments von B-Baum-Datenstrukturen.

F�r die Verwendung des EASy-DSbuilders ist eine Initialisierung seitens eines Lehrenden n�tig. In der alten Version konnten die in Kapitel \ref{sec:funktionalitaet} beschriebenen Eigenschaften �bergeben werden. Es konnte jedoch noch nicht zwischen Datenstrukturen gew�hlt werden. So wurde in der alten Version als Datenstrukturtyp defaultm��ig \textit{Tree} �bergeben. An dieser Stelle musste eine Auswahlm�glichkeit f�r Datenstrukturen in der Erstellungsmaske des Moduls implementiert werden. Die Auswahlm�glichkeit wurde als Dropdown-Men� umgesetzt, wobei die in der Datei \pfile{dsbaStructureType.php} definierten Konstanten als Datenstrukturtypen zur Auswahl gestellt werden.

Nachdem eine Instanz des EASy-DSBuilders erzeugt wurde, muss dem Modul jedoch zur Durchf�hrung eines Assessments eine Datenstruktur seitens des Daten- strukturverarbeitungs-Webservices bereitgestellt werden. Die zur Kommunikation zwischen Moodlemodul und Webservice verwendete Datenstruktur wird in Abschnitt \ref{sec:datenstruktur} vorgestellt. An der Funktionalit�t des Datenstrukturverarbeitungs-Webservice sollte nichts ge�ndert werden. Fehler, die w�hrend der Entwicklung auftraten, wurden behoben. Zur Endwicklung des Moodlemoduls musste jedoch die Datenstruktur eines B-Baums in Java implementiert werden, damit die Funktionen des Webservices bereitgestellt werden konnten. \textbf{�ber die B-Baum Implementierung schreiben?}

Die wichtigen Erweiterungen zur Bereitstellung der Funktionalit�t eines Assessments von B-Baum-Datenstrukturen wurde im Moodlemodul fronend- und backendseitig implementiert. Hierbei steht frontendseitig die Erweiterung des Grapheneditors im Vordergrund, welche ausf�hrlich im Abschnitt \ref{sec:erweiterungEditor} erl�utert wird. Des Weiteren war die Implementierung einer Verarbeitung der Datenstruktur n�tig, sodass zwischen der vom Webservice stammenden Datenstruktur und der f�r den Editor gebrachten Datenstruktur gewandelt werden konnte. Diese Funktionalit�t wird ausf�hrlicher in Abschnitt \ref{sec:strukturverarbeitung} erl�utert. 
Des weiteren wurden noch folgende kleinere Ver�nderungen vorgenommen. In der Datei \pfile{renderer.php}, in der die Ansicht generiert wird, wurde eine Fallentscheidung f�r die Anzeige des Grapheneditors implementiert. Dies ist damit begr�ndet, dass der B-Baum-Editor, wie in Kapitel \ref{sec:konzeptGrapheneditor} begr�ndet, zwei �bereinander liegende jsdot-Editoren ben�tigt. Die �berlagerung der zwei Editoren wurde mit \textit{CSS} umgesetzt.
Weiterhin wurde die AJAX-Api angepasst. Die minimale �nderung bestand darin, dass eine weitere Information �bergeben wird, auf die der B-Baum-Editor aufbaut. N�heres dazu ist in Abschnitt \ref{sec:editorImplemenierungsdetails} zu finden. 

\subsection{Datenstruktur f�r die \\Moodlemodul-Webserice-Kommunikation}
\label{sec:datenstruktur}
Die Datenstruktur, die in in der Kommunikation zwischen Moodlemodul und Webservice eingesetzt wird, spielt eine fundamentale Rolle in der Gesamtanwendung EASy-DSBuilder. Diese Datenstruktur muss in einem JSON-kompatiblen Format alle notwendigen Informationen �ber die zu verarbeitende Baumstruktur beinhalten. Im Falle des B-Baums muss die Datenstruktur als zentrales Element den Knoten mit der jeweiligen Anzahl an Schl�sseln wiedergeben. Des weiteren muss die Datenstruktur noch die Kindsknoten der Wurzelknotens beinhalten, die wiederum richtig positioniert werden m�ssen. So k�nnen die Kindsnoten als linkes oder rechtes Kind des Wurzelknotens positioniert werden. Weiterhin gibt es noch Kindsknoten, die zwischen den einzelnen Schl�sseln des Wurzelknotens positioniert werden (vergl. Kapitel \ref{sec:bbaum}). 

Auf Basis dieser Grundlage wurde zur Abbildung eines B-Baums ein Array als Datenstruktur ausgew�hlt. Dieser Array ist bei einem B-Baum \textit{T} mit einer H�he gr��er eins wie folgt aufgebaut. Erstes Element des Arrays, welches die oberste Wurzel des B-Baums \textit{T} repr�sentiert, ist der linke Kindsknoten der Wurzel. Anschlie�end folgt der erste Schl�ssel der Wurzel, dem ein weiterer Kindsknoten nachfolgt. Diese Abfolge aus Sch�ssel und Kindsknoten erfolgt so lange, bis das Array mit dem rechten Kindsknoten der Wurzel abschlie�t. Die jeweiligen Kindsknoten sind nach der selben Struktur des Wurzelknotens aufgebaut. Eine m�gliche Repr�sentation eines B-Baums durch diese Datenstruktur kann wie folgt aussehen:
\begin{quote}
$[[["16","19"],"31",["37","41"]],"50",[["56"],"86",["96"]]]$
\end{quote} 

Die so eben gezeigte Datenstruktur zeigt einen B-Baum der H�he drei. Der oberste Wurzelknoten enth�lt den Sch�ssel $"3"$. Die Wurzel hat das linke Kind $[["16","19"],"31",["37","41"]]$ und das rechte Kind $[["56"],"86",["96"]]$, welche jeweils separat betrachte einzelne B-B�ume darstellen. Diese Beispiel verdeutlicht anschaulich den rekursiven Aufbau dieser Datenstruktur. Abgeschlossenen wird die Struktur durch kinderlose Knoten, wie sie im vorgestellten B-Baum beispielsweise durch den Knoten $["16","19"]$ repr�sentiert werden. Diese Datenstruktur steht wiederum in einem weiteren Array mit der L�nge zwei an erster Position. An zweiter Position wird die Ordnungszahl \textit{k} (vergl. Kapitel \ref{sec:bbaum}) abgelegt.


\subsection{Datenstrukturverarbeitung backendseitig}
\label{sec:strukturverarbeitung}

Die in diesem Abschnitt vorgestellten �nderungen beziehen sich auf die in Kapitel \ref{sec:dsbuilderbackend} unter Abschnitt \textit{Datenverarbeitung} vorgestellte Funktionalit�t. Bei der Datei, in der �nderungen vorgenommen wurden handelt es sich um die \pfile{lib/js\_dot\_ convert.php}. In ihr befinden sich in der alten Version des EASy-DSBuilders die Klassen \pfile{jsdot\_graph}, \pfile{jsdot\_edge}, \pfile{jsdot\_node} und \pfile{dsbuilder\_binary\_tree}. Die Klasse \pfile{jsdot\_graph} dient als Schnittstelle zum restlichen Modul und verwaltet den jsdot-Graph. Zum Umgang mit �bergebenen Datenstrukturen bedient die Klasse sich im Fall von Bin�rb�umen der Klasse \pfile{dsbuilder\_binary\_tree}. Die nun vorgenommene Erweiterung enth�lt die Implementierung der Klasse \pfile{dsbuilder\_b\_tree}. Diese neu implementierte Klasse orientiert sich stark an der Klasse \pfile{dsbuilder\_ binary\_tree}. Sie stellt die selben Funktionen bereit,  nur dass die Funktionen in der Klasse \pfile{dsbuilder\_b\_tree} dem Umgang mit B-Baum-Datenstrukturen dienen.

Die drei �ffentlichen Funktionen der Klasse \pfile{dsbuilder\_b\_tree} sind die statischen Funktionen 
\textit{initialize\_with\_b\_tree\_data\_structure(...)}, \textit{get\_b\_tree\_data\_ structure(...)} und \textit{compare\_sample\_and\_student\_graph(...)}. Hierbei erm�glichen die ersten beiden Funktionen die Umwandlung der Datenstruktur, sodass mit der Datenstruktur in den beiden Anwendungsf�llen Grapheneditor und Webservices gearbeitet werden kann. Die dritte Funktion vergleicht zwei Graphen und markiert die sich unterscheidenden Knoten. Sie wird f�r die Feedbackfunktion genutzt. Im folgenden werden die Funktionen tiefgehender vorgestellt.

Die Funktion \textit{initialize\_with\_b\_tree\_data\_structure(...)} initialisiert einen neuen \pfile{jsdot\_graph} anhand der in Kapitel \ref{sec:datenstruktur} vorgestellten serialisierten Datenstruktur. Der Funktion k�nnen drei Paramieter �bergeben werden. Die �bergabe einer serialisierte Datenstruktur ist verpflichtend. Ansonsten kann �bergeben werden, wie viel Abstand der Baum im Editor auf der x-Achse zum linken Rand und auf der y-Achse zum oberen Rand haben soll. Standardm��ig werden 50 Pixel f�r die x-Achse und 25 Pixel f�r die y-Achse �bergeben. Jedoch handelt es sich bei der Funktion \textit{initialize\_with\_b\_tree\_data\_structure(...)} nur um eine Wrapperfunktion. Die eigentliche Umwandlung einer serialisierten Datenstruktur in einen \pfile{jsdot\_graph} geschieht in der Hilfsfunktion \textit{b\_tree\_initialize\_helper(...)}. Quellcode \ref{code:bTreeInitHelper} zeigt einen Ausschnitt der Hilfsfunktion.
\begin{figure}[htbp] 
\lstinputlisting[language=PHP, caption=Ausschnitt \textit{b\_tree\_initialize\_helper(...)}, frame=single, label=code:bTreeInitHelper]{code/bTreeInitHelper.php}
\end{figure}
Da es sich bei der serialisierten Datenstruktur um eine rekursiv aufgebaute Datenstruktur handelt ist die Verarbeitung dieser Struktur ebenfalls rekursiv aufgebaut. Der Hilfsfunktion wird eine serialisierte B-Baum-Datenstruktur �bergeben. Wie in Kapitel \ref{sec:datenstruktur} beschrieben besteht die Datenstruktur einem Array, welches Schl�ssel und Kindsknoten-Arrays des Wurzelknotens enth�lt. Auf Grund dieser Struktur wird jedes Element des Wurzelknoten-Arrays durchlaufen (Z. 4). Hierbei werden die einzelnen Knoten-Arrays des �bergebenen B-Baums durch die Variable \pfile{\$json\_row} widergespiegelt. F�r jedes Element in dem jeweiligen Array erfolgt eine Fallunterscheidung, in der gepr�ft wird, ob es sich bei dem jeweiligen Element um ein \textit{String} handelt. Ist das jeweilige Element ein \textit{String} (Z. 5), so muss es sich um einen Schl�ssel im Array handeln. In diesem Fall wird ein neues \pfile{jsdot\_node} initialisiert und den Knoten des \pfile{\$jsdot\_graph} �bergeben (Z. 6 f.). Es werden hierbei jeweils Name, Schl�ssel und Position an den Knoten �bergeben. Wie die Position bestimmt wird, wird in Kapitel \ref{sec:positionierung} erl�utert. Handelt es sich nicht um einen \textit{String}, so muss es sich um ein Kindsknoten-Array handeln. In diesem Fall ruft die Hilfsfunktion sich selber auf, damit die Kindsknoten-Arrays auf die selbe Weise verarbeitet werden k�nnen. Quellcode \ref{code:rekursiverAufrufInitHelper} zeigt den rekursiven Aufruf in seiner Gesamtheit. 
\begin{figure}[htbp] 
\lstinputlisting[language=PHP, caption=rekursiver Aufruf aus Quellcode \ref{code:bTreeInitHelper} in Z. 10, frame=single, label=code:rekursiverAufrufInitHelper]{code/rekursiverAufrufInitHelper.php}
\end{figure}
Der erste �bergebene Parameter ist wieder der in Funktion \textit{initialize\_with\_b\_tree\_data\_structure(...)} initialisierte \pfile{jsdot\_graph}. Der zweite Parameter ist das Kindsknoten-Array. Die n�chsten beiden Parameter beschreiben die Position des Arrays. Als erstes wird die H�he �bergeben, die eine gr��er als die momentane H�he ist. Danach wird die Position innerhalb einer Schicht weitergegeben. \pfile{\$index\_node} sagt dabei aus, um den wievielten Kindsknoten es sich handelt. \pfile{\$indes\_array} zeigt die Position des Wurzelknotens innerhalb der Schicht auf. Diese muss noch mit der Ordnungszahl \pfile{\$t} multipliziert werden, damit die richtige Position f�r den Kindsknoten gefunden wird. Die Summe bildet schlie�lich die Finale Position. Die restlichen Parameter werden unver�ndert weitergegeben.
Zum Abbruch der Funktion dienen die kinderlosen Knoten, da in ihnen keine Kindsknoten-Arrays zum weiteren rekursiven Aufruf vorhanden sind.

Die Funktion \textit{get\_b\_tree\_data\_ structure(...)} wandelt wiederum eine \pfile{jsdot\_graph}-Datenstruktur in eine serialisierte Datenstruktur f�r den Webservice um. Quellcode \ref{code:getBTreeDataStructure} zeigt die gesamte Funktion. 
\begin{figure}[htbp] 
\lstinputlisting[language=PHP, caption=\textit{get\_b\_tree\_data\_ structure(...)}, frame=single, label=code:getBTreeDataStructure]{code/getBTreeDataStructure.php}
\end{figure}
So wird der \pfile{jsdot\_graph} als erstes in die Form \glqq Knoten pro Schicht\grqq umgewandelt (Z. 2). Da diese Form bereits aus Kapitel \ref{sec:editorImplemenierungsdetails} bekannt ist, und die Umwandlung hier �quivalent verl�uft, wird an dieser Stelle nicht weiter darauf eingegangen. Wichtig f�r diese Funktion ist jedoch noch, dass in ihr die Richtigkeit der Syntax des B-Baums gepr�ft wird. Es wird gepr�ft, ob alle Schl�ssel Teil des Baums sind, ob alle Knoten die maximale Anzahl an Schl�sseln nicht �berschreiten und ob der Baum ausbalanciert ist. Anschlie�end wird die rekursive Hilfsfunktion \textit{b\_tree\_structure\_helper(...)} aufgerufen, die die eigentliche Umwandlung umsetzt. Die Hilfsfunktion wird mit dem Wurzelknoten des B-Baums initialisiert (\pfile{\$nodesPerArray[0][0]}). Weiterhin wird noch der gesamte B-Baum, die Position des Wurzelknotens und die Ordnungszahl \pfile{\$t} �bergeben. Quellcode \ref{code:bTreeStructureHelper} zeigt die Hilfsfunktion. 
\begin{figure}[htbp] 
\lstinputlisting[language=PHP, caption=\textit{b\_tree\_structure\_helper(...)}, frame=single, label=code:bTreeStructureHelper]{code/bTreeStructureHelper.php}
\end{figure}
In der Funktion wird jeder Schl�ssel des �bergebenden Knotens (\pfile{\$nodeList}) durchlaufen (Z. 4). F�r jedes Element wird zu erst gepr�ft, ob ein linker Kindsknoten existiert (Z. 5). Ist dies der Fall, ruft die Funktion sich selbst auf, wobei der Kindsknoten als zu bearbeitendes Element �bergeben wird (Z. 6). Anschlie�end wird das Ergebnis dieses rekursiven Aufrufs zum Ergebnis hinzugef�gt (Z. 7). Auch jeder Schl�ssel wird dem Ergebnis �bergeben (Z. 9). Die zweite if-Abfrage (Z. 10) pr�ft, ob sich um den letzten Schl�ssel eines Knotens handelt, und ob dieser einen rechten Kindsknoten hat. Ist dies der Fall, wird f�r diesen Kindsknoten ein weiterer rekursiver Aufruf initiiert (Z. 11). Als abbrechende Bedingung f�r die Rekursion dienen die Kinderlosen Knoten.

\subsection{Grapheneditor}
\label{sec:erweiterungEditor}
Dieses Kapitel erl�utert die neu entwickelte Funktionalit�t des Frontends. Insbesondere hei�t das, dass auf die Umsetzung des Grapheneditors f�r die B-Baumstruktur eingegangen wird. Hierzu wird zu erst die Umsetzung des in Kapitel \ref{sec:konzeptGrapheneditor} vorgestellten Konzepts des Grapheneditors in Kapitel \ref{sec:editorFunktionalitaet} vorgestellt. Anschie�end wird in Kapitel \ref{sec:editorImplemenierungsdetails} tiefgehender auf einzelne, in Kapitel \ref{sec:editorFunktionalitaet} vorgestellten Funktionen eingegangen.
 
\subsubsection{Funktionalit�t}
\label{sec:editorFunktionalitaet} 
Aus dem im Kapitel \ref{sec:konzeptGrapheneditor} bereits erl�uterten Gr�nden soll der Grapheneditor aus zwei separaten Elementen bestehen. Diese beiden Elemente sind zwei jsdot-Editoren, welche �bereinander liegen. Der im Hintergrund liegende Editor ist nicht bearbeitbar und und liefert die B-Baumstruktur, auf der die Schl�ssel des B-Baums angeordnet werden k�nnen. Die Schl�ssel werden von einem im Vordergrund liegendem jsdot-Editor zur Verf�gung gestellt und sind verschiebbar. Um eine h�here Interaktivit�t bereitzustellen entwickelt sich die im Hintergrund liegend Baumstruktur dynamisch. Das bedeutet, dass sobald ein Schl�ssel zu der Baumstruktur hinzugef�gt wird, sich die Baumstruktur automatisch um neue Knoten vergr��ert.  Die Anpassung beinhaltet die Erweiterung des Knotens, in den eingef�gt wurde, sowie die Kindsknoten, die der erweiterte Knoten bereitgestellt bekommt. Ebenso reduziert sich die Baumstruktur nach Endnahme eines Schl�ssels um die jeweiligen Knoten. Sobald ein Schl�ssel auf einen Knoten abgelegt wird, erkennt das System dies und richtet den gesamten Knoten mit dem neu eingef�gten Schl�ssel in Echtzeit aus. Durch die automatische Neuausrichtung der Schl�ssel kann der Studierende erkennen, ob das System seine Eingabe erkannt hat. 

Abbildung \ref{fig:petri} zeigt ein Petri-Netz, welches die Funktionalit�t der Echtzeitverarbeitung beschreibt. Das Petri-Netz beinhaltet drei Arten von Transaktionen. Zur erste Art von Transaktionen geh�ren Referenzen auf Funktionen, die eine Kommunikation zwischen dem GUI und dem Nutzer erm�glichen. Es wird hierf�r auf die jQuerey-Funktionen \textit{mousemove()}, \textit{mousedown()} und \textit{mouseup} verwiesen. Zur besseren Erkennbarkeit sind auf diese Funktionen verweisenden Referenzen im Petri-Netz kursiv abgebildet. Bei der zweiten Art von Transaktionen handelt es sich um Ergebnisse von Verzweigungen. Die dritte Art von Transaktionen beschreibt umfangreichere Funktionalit�ten, welche in Kapitel \ref{sec:editorImplemenierungsdetails} tiefgehender beschreiben werden.

Ausgangspunkt ist die Funktion \textit{b\_tree\_helper()}, welche die Echtzeitverarbeitung initialisiert.  Aus der auf diese Funktion referierenden Transaktion heraus entstehen zwei Stellen, welche auf Eingaben des Nutzers warten. 
\begin{figure}[tbp] 
  \centering
     \includegraphics[width=1\textwidth]{graphics/petriNetz.jpg}
  \caption{Petri-Netz des Grapheneditors}
  \label{fig:petri}
\end{figure}
Es handelt sich hierbei um die Funktionen \textit{mousemove()} und \textit{mousedown()}. \textit{Mousemove()} schl�gt jedes mal aus, wenn der Mauszeiger �ber den Editor f�hrt. Aus der Stelle \textit{Mouse selected} heraus gibt es zwei m�gliche Folgezust�nde. Wurde kein Schl�ssel bewegt (\textit{No key moved}) wird der n�chste Zustand mit der erneuten Belegung der Stelle \textit{Waiting for mousemove()} erreicht. F�r die Transaktion \textit{One key is moved} muss jedoch die Stelle \textit{Key is selected} belegt sein. F�r die Belegung dieser Stelle muss zuerst die Transaktion \textit{mousedown()} durchlaufen sein. Anschlie�end k�nnen die Transaktionen \textit{Clicked on no key} und \textit{Clicked on Key} ausgef�hrt. Der Grund f�r den direkten R�cksprung zur Stelle \textit{Waiting for mousedown()} �ber die Transaktion \textit{Clicked on no key} erfolgt gegen Ende des Kapitels. Wurde jedoch auf einen Schl�ssel gedr�ckt (\textit{Clicked on key}), wird im  Anschluss die Verzweigung durchlaufen, die zwischen dem Schl�ssel auf (\textit{Key is on node}) oder nicht auf einem Knoten (\textit{Key is not on node}) unterscheidet. Ist der Schl�ssel nicht auf einem Knoten, wird direkt die Stelle \textit{Key is selected} markiert. Ist der Schl�ssel hingegen auf einem Knoten wird zuerst die Stelle \textit{changeNode is to be set} markiert. W�hrenddessen wird die Transaktion \textit{Delete key from node and restructure} durchlaufen, sodass die Stelle \textit{Key is selected} schlie�lich auch �ber diese Verzweigung markiert wurde.


In folgenden Absatz wird das Teilnetz um die Stellen \textit{changedNode is to be set} und \textit{changedNode} vorgestellt. Bei \textit{changedNode} handelt es sich um eine Variable, die den zuletzt ge�nderten Schl�ssel beinhaltet. Enth�lt diese Variable einen Wert, dann ist die gleichnamige Stelle im Petri-Netz markiert. Um die Stelle zu markieren k�nnen zwei Transaktionen durchlaufen werden. Zum einen gibt es die Transaktion \textit{Initialize changedNode}, zum anderen \textit{Actualize changedNode}, wobei bei dieser \textit{changedNode} bereits markiert sein muss. Die f�r beide Transaktionen ausgehende Stelle ist \textit{changedNode is to be set}. Die zur Markierung dieser Stelle notwendigen Transaktionen sind \textit{One key is moved} und \textit{Key is on a node}. Dies bedeutet, dass die Variabel \textit{changeNode} gesetzt wird, sobald ein Schl�ssel, der auf einem Knoten liegt, angeklickt, bzw. ausgew�hlt wird. Des weiteren wird die Variable gesetzt, sobald ein ausgew�hlter Knoten -- siehe Abh�ngigkeit zwischen \textit{One key is moved} und \textit{Key is selected} -- auf dem Editor bewegt wird.

Eine von der Stelle \textit{Key is selected} n�chstm�gliche Transaktion ist die Transaktion \textit{mouseup}, welche auf die jQuery-Funktion \textit{mouseup()} referiert. Diese Funktion wird ausgel�st, sobald der Nutzer die Maustaste l�st. Das Petri-Netz zeigt, dass die beiden Stellen \textit{changedNode} und \textit{Key is selected} markiert sein m�ssen, damit die Transaktion \textit{mouseup} durchlaufen werden kann. Insbesondere muss in diesem Fall zwischen der durch die Transaktion \textit{mouseup} repr�sentierten Funktionalit�t und der jQuery-Funkion \textit{mouseup()} differenziert werden. Die jQuery-Funktion \textit{mouseup()} wird ausgel�st, sobald der Nutzer die Maustaste l�st, folglich auch, wenn kein Schl�ssel ausgew�hlt ist. Der auftretende Widerspruch, der durch die Forderung des Petri-Netz nach einen gew�hlten Knoten f�r die Ausf�hrung der Transaktion auftritt, wird  durch Betrachtung der in der Funktion \textit{mouseup()} steckenden, durch die Transaktion beschriebenen, Funktionalit�t aufgel�st. Die durch die Transaktion \textit{mouseup} beschriebene Funktionalit�t enth�lt die notwendige Bedingung, dass ein ver�nderter Schl�ssel vorhanden sein muss. Wird diese Bedingung nicht erf�llt, wird die Funktion abgebrochen. Auf Basis dieser Argumentation ist die durch die Transaktion \textit{mouseup} beschriebene Funktionalit�t nur mit dem Vorhandensein einer \textit{changedNode} verf�gbar, obwohl die jQuery-Funktion \textit{mousemove()} ausgel�st worden ist. Hiermit ist ebenso das Enden der Transaktion \textit{Clicked on no Key} in der Stelle \textit{Waiting for mousedown()} begr�ndet. Hiernach wird nach dem l�sen der Maustaste zwar die jQuery-Funktion \textit{mouseup()} ausgel�st, die aus der Funktion resultierende Funktionalit�t wird jedoch nicht angesprochen.




\subsubsection{Implementierungsdetails}
\label{sec:editorImplemenierungsdetails}

In diesem Kapitel wird die Umsetzung der in Kapitel \ref{sec:editorFunktionalitaet} beschriebene Funktionalit�t der Echtzeitverarbeitung des Grapheneditors vorgestellt. Hierf�r wird zuerst auf die wichtigen Datenstrukturen eingegangen. Anschlie�end werden die f�r die Gesamtfunktionalit�t wichtigen Teilfunktionalit�ten veranschaulicht. Zu diesem Zweck wird insbesondere auf die bereits in Kapitel \ref{sec:editorFunktionalitaet} vorgestellten Teilfunktionalit�ten eingegangen.

Die zentrale Datenstruktur f�r die Echtzeitverarbeitung der Schl�sselausrichtung liefert das mehrdimensionales Array \pfile{nodesPerTier}. Der Datenstruktur kann man die Knoten einer Schicht entnehmen, wobei die einzelnen Knoten wiederum �ber ihre jeweiligen Schl�ssel verf�gen. Alle Knoten einer Schicht meint in diesem Zusammenhang alle Knoten einer bestimmten H�he. So verwaltet die h�chste Arraydimension die Schichten des B-Baums. Insgesamt ist der mehrdimensionale Array so aufgebaut, dass die oberste Schicht des B-Baums auf Position null in dem Array der obersten Dimension abgebildet wird. Die zweite Dimension des Arrays beinhaltet die Knoten einer Schicht. Das Arrays auf Position Null beinhaltet diesbez�glich den Wurzelknoten. Auf Position Eins beinhaltet der Array der ersten Dimension die n�chste Schicht, sodass dort ein Array mit den Kindsknoten der Wurzel  abgelegt ist. Die weiteren Positionen enthalten demnach folglich die weiteren Schichten. 
Die einzelnen Knoten werden wiederum auch durch Arrays dargestellt, welche alle Schl�ssel eines Knotens enthalten. Somit handelt es sich bei dem Array \pfile{nodesPerTier} um ein dreidimensionales Array mit Schichten auf der ersten, Knoten auf der zweiten und Schl�sseln auf der dritten Dimension. 

Eine weitere wichtige Datenstruktur ist das Array \pfile{nodeListBounds}. Diese Datenstruktur enth�lt die Grenzen, in denen ein Schl�ssel als Schl�ssel innerhalb eines Knotens erkannt wird. Wird folglich ein Sch�ssel auf dem Editor innerhalb dieser Grenzen abgelegt, wird ein Positionierungsprozess angesto�en, in dem die Schl�ssel eines Knotens neu positioniert werden. Das Array \pfile{nodeListBounds} ist dem Array \pfile{nodesPerTier} stark nachempfunden. So handelt es sich ebenfalls um ein dreidimensionales Array, wobei auch hier die Schichten auf der ersten und die Knoten auf der zweiten Dimension abgelegt werden. In der dritten Dimension sind jedoch anstatt von Schl�sseln die Grenzen der einzelnen Knoten gespeichert. 


Die Funktion \pfile{b\_tree\_helper} ist die Hilfsfunktion, welche eine Echtzeitverarbeitung der Nutzereingabe erm�glicht. Sie bekommt als Parameter die beiden Graphen, den Vordergrund- und den Hintergrundgraphen, und die Ordnung \textit{t} �bergeben.


\subsubsection{Aufbau des B-Baums im Hintergrund}
\label{sec:aufbauBBaumHintergrund}
Dieser Abschnitt erl�utert das Prinzip nach dem die B-Baum-Struktur im Hintergrund des Grapheneditors aufgebaut wird. 

\section{Fazit und Ausblick}

In dieser Arbeit wurde die Erweiterung des Moodle-Moduls EASy-DSBuilders um das Assessment von B-Baum-Datenstrukturen dokumentiert. 
Hierzu wurde eingangs der EASy-DSBuilder in seiner Struktur und Funktionalit�t vorgestellt. Anschlie�end folge eine Anforderungsanalyse zur Erweiterung des Moduls um die Funktionalit�t eines Assessments von B-Baum-Datenstrukturen. 
Darauf folgend wurden die Stellen im Modul vorgestellt, an denen Modifikation und Erweiterungen durchzuf�hren waren, um die auf Basis der Analyse erarbeiteten Anforderungen umsetzen zu k�nnen. Als Ergebnis kann ein funktionsf�higes Modul in seiner zweiten Version pr�sentiert werden, welches das Assessment von B-Baum-Datenstrukturen unterst�tzt. 

So kann der Lehrende in dieser zweiten Version des EASy-DSBuilders bei der Initialisierung einer neuen Instanz dieses Moduls zwischen den zu pr�fenden Datenstrukturen w�hlen. 
Dem Studierenden wurde hingegen ein erweiterter Editor bereitgestellt, der f�r das Assessment von B-B�umen optimiert ist. Dem Studierenden wird die M�glichkeit geboten, die auf dem Editor befindlichen Schl�ssel intuitiv zu bewegen und auf einer B-Baum-Grundstruktur im Hintergrund, welche sich weiterhin automatisch anpasst, abzulegen. Dabei wurden s�mtliche Funktionalit�ten des Moduls der ersten Version beibehalten und auf das neue Assessment �bertragen. 
Hierzu geh�rt die M�glichkeit, im zehnschrittigen Assessment durch die einzelnen Schritte zu navigieren. 
Eine weitere wichtige Funktion ist die des Feedbacks. 
So kann der Lehrende auch f�r das Assessment von B-B�umen ein Feedback einstellen, welches dem Studierenden Informationen dar�ber gibt, welche Fehler er bei seiner Eingabe gemacht hat.


Eine M�glichkeit der Verbesserung des Moduls liegt in der Optimierung der Positionierung der Elemente eines B-Baums im Editor.
In der gegenw�rtigen Version wird jedem Element ein fester Platz zugeordnet. So bleiben auf der horizontalen Achse Freir�ume zwischen einzelnen Elementen, wenn eine Schicht nicht voll belegt ist. 
Auf Grund des exponentiellen Wachstums und der vollen Ausgeglichenheit der H�he braucht der B-Baum insbesondere bei h�herer Ordnung \textit{m} so viel Platz auf der x-Achse, dass die Usability eingeschr�nkt wird.
Hier sind die unn�tigen Zwischenr�ume insbesondere in der Blattschicht zu eliminieren, um den Gesamtbaum zu Verschlanken.
\section*{Anhang}
\label{sec:anhang}


\lstinputlisting[language=Java, caption=Aufruf zur Initialisierung eines JSDot-Graphs, frame=single, label=code:wrapper]{code/DataStructureWrapperInterface.java}


 \bibliography{Bachelorarbeit}
  \bibliographystyle{alpha}
 

  \diplomabschlusserklaerung{(Abgabedatum)}
\end{document}
