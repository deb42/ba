\documentclass[a4paper,12pt]{article}
\usepackage[latin1]{inputenc}
\usepackage[pdftex]{color,graphicx}
\usepackage[hypertexnames=false]{hyperref} 
\usepackage[german,ngerman]{babel}
\usepackage{fancyhdr}
\usepackage{amssymb}
\usepackage{background}
\usepackage{amsmath}
\usepackage[rflt]{floatflt}
\usepackage{tabularx}
\usepackage{ausarbeitung}

%% Diese Farben werden f�r den Quelltext verwendet
\definecolor{srcblue}{rgb}{0,0,0.5}
\definecolor{srcgray}{rgb}{0.5,0.5,0.5}
\definecolor{srcred}{rgb}{0.5,0,0}


%% Diese Zeile unbedingt stehen lassen und anpassen - sie enth�lt Autor und Titel der Ausarbeitung
\mywork{David Bujok}{Thema der Arbeit}

\begin{document}
\SetBgContents{}
	%% Bei Diplomarbeiten folgende Zeile nutzen
	\mybachelortitle{Dipl.-Wirt.Inform. Claus Alexander Usener}
	%% Bei Bachelorarbeiten diese Zeile auskommentieren
	%%\mybachelortitle{(ggf. Name des Betreuers)}
	%% Bei Seminararbeiten diese Zeile auskommentieren
	%%\myseminartitle{Titel des Seminars}{(ggf. Name des Betreuers)}

  %% Inhaltsverzeichnis
  %% frontmatter setzt die Seitenzahlen auf i, ii, ...
  \frontmatter

	%% Generiert automatisch aus den Sektionsbefehlen ein Inhaltsverzeichnis  
  \tableofcontents

  \newpage

	%% das Mainmatter sorgt f�r die Nutzung arabischer Seitenzahlen
	\mainmatter
	
\section{Einleitung}
\label{sec:einleitung}

\glqq 
E-Assessment ist eines der Schlagworte der letzten Jahre\grqq \ \cite[S. 11]{Ruedel2010}.
Unter dieser Form des Assessments versteht man das Nutzen von Hardware und Software im Pr�fungsprozess. 
So wurden im deutschsprachigem Raum f�r E-Assessment Synonyme wie Online-Pr�fungen oder computergest�tzte Pr�fungen verwendet \cite[S. 13]{Ruedel2010}.
Zwei Gr�nde sprechen f�r diese Assessmentform als Mittel des Leistungsnachweises im Pr�fungsalltag.
Durch Effizienzsteigerung kann knappen Raum- und Personalressourcen vorbeugen. 
Au�erdem k�nnen neue didaktische Konzepte wie bessere Visualisierung oder neue Aufgabenformate, die Computer erst erm�glichen, angeboten werden \cite[S. 11 ff.]{Ruedel2010}. 
Die meisten Systeme, die E-Assessments anbieten, stellen jedoch nur einfache Aufgabentypen wie Multiple-Choice-Aufgaben zur Verf�gung. Da insbesondere technische Disziplinen wie die Informatik auf Aufgabentypen zur Erhebung von kognitiven F�higkeiten angewiesen sind \cite[S. 24]{Kuchen2010}, 
%Die steigende Anzahl von Studierenden bei gleichbleibender Zahl der Lehrpersonen und Assistierenden stellt Universit�ten vor neue Herausforderungen. Insbesondere die Logistik der steigenden Anzahl von Modulabschl�ssen bei knappen Raum- und Personalressourcen wird immer mehr zum Problem. Hierf�r bietet E-Assessment mit der M�glichkeit der Verbesserung und Erh�hung der Effizienz im Pr�fungsalltag einen m�glichen Ausweg. Unter E-Assessment versteht man das Nutzen von Hardware und Software im Pr�fungsprozess. So wurden im deutschsprachigem Raum f�r E-Assessment Synonyme wie Online-Pr�fungen oder Computergest�tzte Pr�fungen verwendet \cite[S. 11 ff.]{Ruedel2010}.
entwickelte der Lehrstuhl f�r Praktische Informatik in der Wirtschaft der WWU M�nster das E-Assessment-Tool EASy-DSBuilder zum Assessment von AVL-Baum-Datenstrukturen f�r die Moodle-Distribution  Learnweb \cite[S. 1]{Usener2014}. Bei Moodle handelt es sich um ein weltweit stark verbreitetes Lernmanagement-System \cite[S. 33]{Gertrsch2007}, das Lehrenden, Administratoren und Lernenden eine robuste, sichere und integrierte Plattform bereitstellen soll \cite{moo15d}.
%Insbesondere viele bekannte Lernplattformen unterst�tzen den �bungsbetrieb durch Funktionen zu Bereitstellung von Aufgabenbl�ttern und zur Organisation studentischer L�sungen. Einige Systeme bieten dar�ber hinaus eine Elektronische �berpr�fung der L�sung an. \cite{Kuchen2010}.
%Unter Lernplattformen ist Moodle international die am weitesten verbreitete Lernplattform \cite{hei15}. 
%Sie ist weltweit in 231 L�ndern �ber 53.000 Seiten registriert \cite{moo15a}
%Bei Moodle handelt es sich um ein Softwarepaket, welches einen  konstruktivistischen Lehr- und Lernansatz unterst�tzt. \cite{moo15a} 
%Weiterhin ist Moodle eine frei verf�gbare Open Source Software. %Dies bedeutet, dass sie frei erh�ltlich ist. 
Das System bietet die M�glichkeit der individuellen Anpassung an spezifische Anwendungssituationen \cite{moo15a}.
%Auch die Westf�lische Wilhelms-Universit�t M�nster stellt zur Verbesserung des Lehrbetriebs eine Moodledistribution unter dem Namen Learnweb zur Verf�gung.
%F�r die Vorlesung \textit{Informatik I} wurde bereits ein Moodlemodul implementiert, welches die M�glichkeit  bietet ????

Das Ziel dieser Arbeit ist die Entwicklung und Vorstellung eines E-Learning-Moduls zum Assessment von B-Baum-Strukturen.  
Im Zentrum steht hierbei das Moodle-Modul EASy-DSBuilder. Insbesondere stellt diese Arbeit die Erweiterung dieses Moduls um des Assessments von B-Baum-Datenstrukturen vor.
Grundlagen dieser Arbeit (Kapitel \ref{sec:grundlagen}) sind eine Einf�hrung in die Lernplattform Moodle, die Formen des E-Assessments, und schlie�lich die Definition einer B-Baum-Datenstruktur.
Das darauf folgende Kapitel \ref{sec:dsbuilder} stellt das Moodle-Modul EASy-DSBuilder vor. Hierbei wird auf die Funktionalit�t aus Benutzersicht und auf die Struktur aus technischer Sicht eingegangen. Im Kapitel \ref{sec:anforderungen} werden die Anforderungen an ein Tool zum Assessment von B-Baum-Datenstrukturen vorgestellt. Insbesondere werden im Kontext des EASy-DSBuilder �nderungsanforderung an dieses Modul zur Umsetzung des Assessments von B-Baum-Datenstrukturen dargelegt. Anschie�end wird die Umsetzung der Anforderungen betrachtet. Hierbei werden die entwickelten Funktionalit�ten beleuchtet und es wird ein Einblick auf ausgew�hlte Implementierungsdetails gegeben. Abschie�end wir ein Fazit des Ergebnisses dieser Arbeit betrachtet und ein Ausblick auf m�gliche weitere Entwicklungen des Moduls gegeben.


\section{Die Lernplattform Moodle}

\subsection{Was ist Moodle}
Bei Moodle handelt es sich um ein weltweit anerkanntes Lernmanagement-System \cite[S. 33]{Gertrsch2007}, das Lehrenden, Administratoren und Lernenden eine robuste, sichere und integrierte Plattform bereitstellen soll \cite{moo15d}. Der Name Moodle leitet sich von der Akronymisierung des Ausdrucks \textit{\textbf{M}odular \textbf{O}bject \textbf{O}riented \textbf{D}ynamic \textbf{L}earning \textbf{E}nvironment} ab \cite[S. 33]{Gertrsch2007}.
Moodle ist weiterhin eine frei verf�gbares Softwarepaket, da es der GNU Public Lizenz unterliegt \cite{Scheb2009}. Software, welche unter einer GNU Public License vertrieben wird, darf kopiert, benutzt und weiterentwickelt werden. Eine einschr�nkende Bedingung ist, dass �nderungen oder Weiterentwicklungen den eben genannten Pflichten unterliegen, sie folglich auch ver�ffentlicht und Dritten zur Verf�gung gestellt werden m�ssen \cite{moo15a}. Die Plattform wird von einer weltweiten Gemeinschaft und von der Moodle Pty. Ltd. laufend weiterentwickelt. Vom australischen Moodle Erfinder Marign Dougiamas wird das Projekt zielgerichtet geleitet. Des weiteren gibt es ein Netzwerk professioneller Partnerunternehmen, welche Support und Beratung leisten \cite[S. 12]{Scheb2009}.

\subsection{Moodle als Lernmanagement-System}
Unter einem Lernmanagement-System (LMS) versteht man im wesentlichen ein Management-System f�r die Automatisierung und die Administration von Ausbildung. Insbesondere sollten LMS �ber folgende Funktionen verf�gen \cite[S. 14]{Schulmeister2005}:

\begin{itemize}
\item Eine Benutzerverwaltung (Anmeldung mit Verschl�sselung)
\item Eine Kursverwaltung (Kurse, Verwaltung der Inhalte, Dateiverwaltung)
\item Eine Rollen- und Rechtevergabe mit differenzierten Rechten
\item Kommunikationsmethoden (Chat, Foren) und Werkzeuge f�r das Lernen (Whiteboard, Notizbuch, Annotationen, Kalender etc.)
\end{itemize}
Moodle stellt diese Funktionen zur Verf�gung. So besteht �ber die Website-Administration die M�glichkeit der Benutzerverwaltung \cite[S. 563
 2 ff.]{Gertrsch2007} und der Kursverwaltung \cite[S. 588 ff.]{Gertrsch2007}. Bei der Rollen - und Rechtevergabe bietet Moodle flexible M�glichkeiten der Administration. So verf�gt Moodle �ber vorgefertigte Basisrollen mit bestimmten Rechten, die einen Gro�teil der Anwendungsf�lle abdecken. F�r bestimmte Situationen k�nnen Rollen jedoch editiert oder neue Rollen erstellt werden \cite[S. 191]{Gertrsch2007}. Die Basisrollen des Systems sind \cite[S. 193]{Gertrsch2007}:
\begin{itemize}
\item \textit{Kursverwalter:} Wer in einem Kontext \textit{Kursverwalter} ist, kann einen \textit{neuen Kurs erstellen} und in diesem unterrichten, weil er automatisch als \textit{Trainer} eingetragen wird. Zu anderen Kursen im gleichen Kontext hat er aber keinen Zugriff.
\item \textit{Trainer}: Wer in einem Kontext \textit{Trainer} ist, ist in s�mtlichen Kursen dieses Kontextes als \textit{Trainer} eingetragen und kann diese Bearbeiten
\item \textit{Trainer ohne Editorrecht}: ist Trainer in s�mtlichen Kursen dieses Kontextes. 
\item \textit{Teilnehmer/in}: ist Teilnehmer in s�mtlichen Kursen dieses Kontextes, kann also auch Kurse mit Zugriffsschl�ssel betreten.
\end{itemize}
Auf die Kommunikationsmethoden, die Moodle zur Verf�gung stellt, wird in Kapitel \ref{sec:kommunikationsmethoden} eingegangen. 
 

Abbildung \ref{fig:architekurLMS} zeigt die idealtypische Architektur eines LMS. Zu sehen ist, dass ein LMS �ber drei Schichten verf�gt. Bei der untersten Schicht handelt es sich um die Datenbankschicht, in der alle Lernobjekte,Benuterdaten und andere gehalten werden. Die mittlere Schicht stellt Schnittstellen zur Verf�gung. Die oberste Schicht stellt die Sicht bereit, �ber die �ber die seitens von Administratoren, Dozenten oder Studierenden auf Inhalte zugegriffen werden kann \cite[S. 11]{Schulmeister2005}.
\begin{figure}[htbp] 
  \centering
     \includegraphics[width=0.9\textwidth]{graphics/ArchitekturLMS.jpg}
  \caption{Idealtypische Architektur eines LMS \cite[S. 12]{Schulmeister2005}}
  \label{fig:architekurLMS}
\end{figure} 
Im Kontext dieser Arbeit wird insbesondere verst�rkt auf die Schnittstellenschicht eingegangen. Das Kapitel \ref{sec:modularitaet} wird die M�glichkeit Einbindung von Modulen erl�utern. Das Kapitel \ref{} wird hingegen den Teilbereichen Aufgeben und Tests aus dem Bereich Authoring der Ansichtsschicht auseinandersetzen. \textit{Es wird der Forschungsbereich E-Assessment vorgestellt, welcher sich mit �berpr�fungen �ber Onlinemedien auseinandersetzt}.




\subsection{Kommunikationsmethoden in Moodle}
\label{sec:kommunikationsmethoden}



\subsubsection{Aufbau eines Plugins}
\label{sec:aufbauPlugin}

F�r jedes Plugin in Moodle muss eine bestimmte Datenstruktur implementiert werden. Diese besteht aus separaten Unterverzeichnissen und verpflichtenden Dateien. Des weiteren haben Entwickler die M�glichkeit weitere Dateien selbst zu gestalten \cite{moo15b}. 
\hfill \\ \hfill \\ 
\pfile{/<modname>/backup} \hfill \\ 
Dieser Ordner dient zur Ablage aller Dateien, welche definieren, wie sich das Modul bei einem Backup oder einer Wiederherstellung verhalten soll \cite{moo15b}.

\hfill \\ 
\pfile{/<modname>/db}
\begin{itemize} 
	\item \pfile{/access.php} In dieser Datei werden die so genannten \textit{capabilities} f�r das Plugin definiert. \textit{capabilities} beschreiben die Berechtigungen, welche eine Rolle in diesem Plugin zugeordnet bekommt. Eine Berechtigung ist beispielsweise das hinzuf�gen einer neuen Instanz diese Plugins zu einem Kurs \cite{moo15b}.
	\item \pfile{/install.xml} Diese Datei wird bei der Installation des Moduls benutzt. Sie definiert, welche Datenbanktabellen und -felder erstellt werden. Hierf�r wird das XML-Format verwendet. Braucht das Modul keine weiteren Tabellen oder Spalten, so kann auf diese Datei verzichtet werden \cite{moo15b}.
	\item \pfile{upgrade.php} Auf Grund dessen, dass die Datei \pfile{install.xml} nur einmal w�hrend der Installation aufgef�hrt wird, braucht es eine Methode um die Datenbank nachtr�glich um Tabellen oder Spalten zu erweitern. Diese Funktionalit�t wird von dieser Datei bereitgestellt und kommt bei einem Update des Moduls zum Einsatz \cite{moo15b}.
\end{itemize}
\hfill \\ 
\pfile{/<modname>/lang} \hfill \\ 
In diesem Ordner k�nnen alle \textit{Strings} gespeichert werden, die im Modul benutzt werden sollen. Jede Sprache hat hierbei einen spezifischen Ordnernamen ('\pfile{/lang/de}' beispielsweise f�r die Sprache Deutsch). Die in diesem Ordner gespeicherte Datei muss in der Form \pfile{<modname>.php} benannt sein \cite{moo15b}.

\hfill \\ 
\pfile{/<modname>/pix} \hfill \\ 
Dieser Ordner dient dazu das Logo des Moduls zu speichern, welches neben dem Modulname erscheint. Der Name des Logos muss \pfile{icon.gif} lauten. Weiterhin besteht die M�glichkeit weitere Bilder in diesem Ordner zu speichern
\cite{moo15b}.

\hfill \\ \hfill \\ 
\pfile{/<modname>}
\begin{itemize} 
	\item \pfile{/lib.php} Diese Datei bietet eine Schnittstelle f�r die zu implementierenden Kernfunktionen. Kernfunktionen werden dazu ben�tigt, damit das Modul in Moodle integriert arbeiten kann. Diese
Schnittstellen-Funktionen werden von Moodle nach einem bestimmten Ereignis im Prozessablauf aufgerufen, sofern diese vom Modul in der Datei \pfile{/lib.php} definiert wurden. Dabei ist jeder dieser Funktionen zun�chst der Name des
Moduls vorangestellt, gefolgt von einem Unterstrich und dem Funktionsnamen
(\pfile{<pluginname>\_core\_function}). Diese Konvention ist deshalb so wichtig, da
die Datei \pfile{/lib.php} keine Klasse definiert, welche Namenskonflikte verhindern
w�rde. \cite{moo15c}
	
	\item \pfile{/mod\_form.php}  Diese Datei wird beim Hinzuf�gen oder Bearbeiten eines Kurses genutzt. Es enth�lt die Elemente welche im Editiermen� des Moduls zu sehen sind. Die in dieser Datei enthaltende Klasse muss der Namenskonvention nach in der Form 
	\pfile{mod\_<modname>\_mod\_form} benannt sein.
	
	\item \pfile{/index.php} Diese Datei wird von Moodle dazu genutzt, um auf Aktivit�ten bei allen Instanzen dieses Moduls, welche einem bestimmten Kurs �bergeben wurden, zu h�ren. Diese Datei ist spezifisch f�r diese Modulart \textit{Activity Module}.
	
	\item \pfile{/view.php} Diese Datei wird bei der Erzeugung der Anzeige ben�tigt. Beim Aufrufen eines Moduls �ber die Kurssicht wird auf diese Datei verwiesen.  Dabei wird dieser Datei die Instanz-ID �bergeben, anhand welcher
die Daten der Instanz ausgew�hlt und angezeigt werden k�nnen. Diese Datei ist spezifisch f�r diese Modulart \textit{Activity Module}.
	
	\item \pfile{/version.php} Diese Datei enth�lt die aktuelle Versionsnummer dieses Moduls. Au�erdem enth�lt diese Datei weitere Attribute wie beispielsweise die Mindestanforderungen hinsichtlich der Moodleplattform.
	
	
\end{itemize}
\section{Vorstellung des Moodleplugins EASy-DSBuilder}
\label{sec:dsbuilder}
Der EASy-DSBuilder ist ein E-Assessent Tool, welches der Evaluation grundlegender Konzepte �ber Operationen (z.B. Suchen, Einf�gen, und Entfernen) innerhalb der Datenstruktur \textit{Bin�rbaum} dient \cite{Usener2014}. 

Das Tool wurde speziell f�r die Lernplattform Moodle implementiert. 

Diese Kapitel wird das Tool EASy-DSBuilder vorstellen. Hierbei wird zu erst in Kapitel \ref{sec:funktionalitaet} auf die Funktionalit�t aus Benutzersicht eingegangen. Anschlie�end erfolgt eine Erl�uterung der Implementation (Kapitel %\ref{?}).

\subsection{Funktionalit�t aus Benutzersicht}
\label{sec:funktionalitaet}

Im folgenden Kapitel wird die Funktionalit�t des Moodleplugins EASy-DSBuilder vorgestellt. Hierbei wird auf die beiden Sichten Student und Lehrender eingegangen.

\paragraph{Lehrender} \hfill \\
Der Lehrende hat zwei Grundlegend Aufgaben. Zum einen ist er daf�r verantwortlich, dass eine Aufgabe erstellt wird, zum anderen hat er die M�glichkeit, die Ergebnisse einzusehen, um beispielsweise Indikatoren zur Verbesserung der Lehre zu finden \cite{Usener2014}. Wird eine neue Aufgabe erstellt, hat der Lehrende die M�glichkeit allgemeine Informationen wie den \textit{Titel}, die \textit{Beschreibung} und das \textit{F�lligkeitsdatum} anzugeben. Unter \textit{Source Files} kann der Lehrende �ber Drag-and-Drop seine eigene Implementierung einer Datenstruktur zu dem Moodleplugin hinzuf�gen. Hierzu muss er jedoch eine Wrapper auf Basis eines Interfaces implementieren, welches die verlangten Voraussetzungen erf�llt. Diese Wrapperklasse muss anschlie�end vom Lehrenden als Hauptklasse eingestellt werden. Auf die Funktionalit�t der Wrapperklasse aus technischer Sicht wird im Kapitel \ref{sec:technWebService} n�her eingegangen. Des weiteren kann der Lehrende eine Feedback aktivieren. Die genau Funktionalit�t des Feedbacks wird im Absatz der Studentensicht erl�utert.  

\paragraph{Studierender} \hfill \\
Der Studierende verf�gt �ber zwei Ansichten. Zum einen die �bersichtsansicht, zum anderen die Bearbeitungsansicht.
Nachdem der Studierende sich in das Plugin eingew�hlt hat, ist ist �bersichtsansicht �ber den bisherigen Verlauf des Assessments zu sehen. In dieser �bersicht ist der Abgabestatus, der Bewertungsstand, der Abgabezeitpunkt und die verbliebene Zeit zu sehen (vergl. Abb. \ref{}). �ber den Button \textit{Aufgabe bearbeiten} gelangt der Studierende zum Editor, in dem die Aufgabe bearbeitet werden kann. 

Die Bearbeitungsansicht ist in drei grundlegende Abschnitte unterteilt. Den oberen Teil der Ansicht bildet ein �berblick �ber den aktuellen Schritt. Dieser �berblick beinhaltet den Fortschritt der Aufgabe, die Nummer des aktuellen Schritts und den aktuellen Arbeitsauftrag. Im mittleren Teil der Sicht befindet sich der Editor, in dem der Studierende die Aufgabe bearbeiten kann. Im oberen linken Bereich des Editor befinden sich drei Kn�pfe (vergl. Abb. \ref{}.1), �ber welche der Editiermodus ausgew�hlt werden kann. Der 1. Knopf erm�glicht das verschieben von Konten im Editor, der zweite Knopf erm�glicht das Ziehen von Kanten zwischen zwei Knoten, und der dritte Knopf erm�glicht das Entfernen von Konten.

Der DragandDropGrafikeditor enth�lt zwei bearbeitbare Elemente, die Knoten und die Kanten. �ber Manipulation dieser Elemente sollen Studierende den
Umgang mit Datenstrukturoperationen erlernen. Hierbei kann der Studierende Operationen wie das Einf�gen in oder das L�schen aus einer Datenstruktur 
praktizieren. In der \textbf{momentanen} Version des EASyDSBuilders beginnt jeder Schritt mit dem Ergebnisbaum des zuvor eingereichten Schrittes oder einem Initiierngsbaum 
wenn, es sich um den  ersten Schritt handelt.
Auf der linken Seite des Editors wird der einzuf�gende Knoten bereitgestellt. Die Aufgabe des Studierenden ist es, diesen Knoten an der richtigen 
Stelle in den Baum einzuf�gen. \textbf{Erl�uterung der M�glichkeiten von Manipulationen}
Nachdem der Studierende seine Ver�nderungen vorgenommen hat, kann er �ber den Knopf \textit{Syntax pr�fen} den Baum ausbalanciert anzeigen lassen. Auf diese Weise kann der Studierende �berpr�fen, ob die Anwendung den Baum im Sinne des Studierenden verarbeitet hat. Entspricht die �berpr�fte Struktur nicht der Struktur eines Baumes, \textbf{genauere Definiton} bekommt der Studierende eine Fehlermeldung mit Hinweis �ber die Fehlerquelle.

Hat der Lehrende bei der Einrichtung des DSBuilders die Option \textit{direktes Feedback} eingestellt, erscheint im Falle einer falschen Eingabe ein Feedbackfeld unterhalb des Editors. In diesem Feedbackfeld wird zu erst ein Informationstext angezeigt, welches das richtige Vorgehen in dem zuvor eingereichtem Schritt beschreibt. Unterhalb dieses Informationstextes ist der korrekte Baum zu sehen. Die falsch eingeordneten Knoten sind rot markiert. 


\subsection{Umsetzung aus technischer Sicht}
\label{sec:technologien}

Das gesamte System um den EASy-DSBuilder besteht backendseitig aus zwei separaten Systemen. Zum einen gibt es das eigentliche Moodleplugin, welches in eine bestehenden Moodelplattform integriert werden kann, zum anderen gibt es einen Datenstruktur-Verarbeitungsservice, welcher als Webservice implementiert ist. Das Moodleplugin hat die M�glichkeit �ber die Moodle-API Daten in einer SQL-Datenbank - beispielsweise einem MySQL-Server - zu hinterlegen. Die Kommunikation zwischen dem Moodleplugin und dem Webservice l�uft �ber das SOAP-Protokoll. Der Webservice ist als WildFly Application Server implementiert und unterliegt somit dem Java-EE7-Standard \cite{wildflyCert}. In Abbildung \ref{fig:technoloieUeberblick} ist dargestellt, wie die unterschiedlichen Technologien in einander greifen.

\begin{figure}[htbp] 
  \centering
     \includegraphics[width=0.9\textwidth]{grahics/UeberblickTechnologien.jpg}
  \caption{Technischer �berblick}
  \label{fig:technoloieUeberblick}
\end{figure}

Der Datenstruktur-Verarbeitungsservice hat die Aufgabe Datenstrukturen mit Hilfe von Die Separierung des Systems erfolgt aus den Risiken, dass der Code sch�dlich sein oder eine schlechte Ausf�hrungsleistung aufweist kann. Durch die Trennung der beiden Systeme kann in beiden F�llen Zusammen- oder Performanceeinbr�chen der gesamten E-Learning-Plattform vorgebeugt werden. Weiterhin kann so Datendiebstahl vorgebeugt werden, da in der Verarbeitungsumgebung keine nutzerbezogenen Daten verarbeitet werden. Bei Ausfall des Verarbeitungsservices ist jedoch das Aufrufen eines n�chsten Schrittes nicht mehr m�glich \cite{Usener2014}.

Auf Clientseite wird HTML mit CSS und JavaScript verwendet, um das Plugin f�r den Benutzer darstellen zu k�nnen. Als JavaScript-Frameworks wird jQuery und und als JavaSrcipt-Applikation wird jsdot eingesetzt. �ber jQuery ist die Kommunikation mit dem Moodleplugin �ber das AJAX-Protokoll organisiert. Jsdot dient als Grapheditor.  

\subsubsection{Datenstruktur-Verarbeitungsservice}  
\label{sec:technWebService}

Der Datenstruktur-Verarbeitungsservice kompiliert und f�hrt den vom Lehrenden bereitgestellten Code aus. Er ist als Webservice implementiert und kann somit von einem anderen Server aus bereitgestellt werden. Die Ausf�hrung des Codes ist vor jedem Einf�gen oder L�schen, das von einem Studierenden durchgef�hrt wird, notwendig.

Auf der Grundlage des bisherigen, eingereichten Schritts berechnet die Ausf�hrungsumgebung den n�chsten  die Ausf�hrungsumgebung der n�chsten Betriebswert (Taste, die eingef�gt oder gel�scht wird), die erwartete L�sung und die entsprechende detaillierte R�ckmeldungen.

\subsubsection{Moodleplugin backendseitig}
\label{sec:dsbuilderbackend}
Das backendseitige Moodleplugin besitzt die grundlegende Struktur eines Moodleplugins, wie sie in Kapitel \ref{sec:aufbauPlugin} dargestellt wurde. 

�ber die Datei \textbf{view.php} wird der EASy-DSBuilder initialisiert. Von hier aus wir die Datei \textbf{renderer.php} angesto�en. Diese Datei sorgt daf�r, dass die Ansicht erstellt wird. 

\textbf{AJAX}
Zur Kommunikation stellt das Moodleplugin eine AJAX Api zur Verf�gung. 

Der Codeausschnitt \ref{code:ajax} zeigt die Aktionen, die nach einer AJAX-Anfrage durchgef�hrt werden k�nnen. Der erste �berpr�ft, ob es sich bei der �bergebenen Struktur um eine richtige handelt

\lstinputlisting[language=PHP, caption=AJAX API, frame=single, label=code:ajax]{code/ajax.php}


%\end{scriptsize}



\subsubsection{Moodleplugin frontendseitig}
\label{sec:dsbuilderfrontend}



 \bibliography{Bachelorarbeit}
  \bibliographystyle{alpha}
 

  \diplomabschlusserklaerung{(Abgabedatum)}
\end{document}
