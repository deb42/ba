\section{Umsetzung der �nderungsanforderungen}
In den folgenden Abschnitten wird die Umsetzung der �nderungsanforderungen beschreiben. Dies bedeutet, dass erl�utert wird, an welchen Stellen Erweiterungen vorgenommen werden mussten, und beschrieben wird, wie diese �nderungen in der Implementierung umgesetzt wurden. 

\subsection{Umbau des EASy-DSBuilders}
Dieses Kapitel stellt die Stellen vor, an denen �nderungen vorgenommen werden mussten und erl�utert die Ursachen, auf Grund derer die �nderungen vorgenommen werden mussten. In Kapitel \ref{sec:implementierungsdetails} wir auf die Implementierungsdetails eingegangen. Dieses Kapitel ist nach Schichten strukturiert.

\subsubsection{Datenstrukturverarbeitungs-Webservice}
An der Funktionalit�t des Datenstrukturverarbeitungs-Webservice sollte nichts ge�ndert werden. Fehler, die w�hrend der Entwicklung auftraten, wurden behoben. Zur Endwicklung des Moodlemoduls musste jedoch die Datenstruktur eines B-Baums in Java implementiert werden, damit die Funktionen des Webservices bereitgestellt werden konnten.

\subsubsection{Moodlemodul backendseitig}
Die f�r das backendseitige Moodlemodul wichtigen Funktionalit�ten ist die in Kapitel \ref{sec:dsbuilderbackend} beschriebene Generierung des DOM-Codes, das Handling von AJAX-Anfragen und das Verarbeiten und zur Verf�gung Stellen der Datenstruktur. Die weitere Gliederung dieses Abschnitts ist an die Gliederung des Kapitels \ref{sec:dsbuilderbackend} angelehnt.

\paragraph{Generierung des DOM-Codes} \hfill \\


\subsection{Ausgew�hlte Implementierungsdetails}
\label{sec:implementierungsdetails}