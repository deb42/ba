\section{Einleitung}
\label{sec:einleitung}

Die steigende Anzahl von Studierenden bei gleichbleibender Zahl der Lehrpersonen und Assistierenden stellt Universit�ten vor neue Herausforderungen. Insbesondere die Logistik der steigenden Anzahl von Modulabschl�ssen bei knappen Raum- und Personalressourcen wird immer mehr zum Problem. Hierf�r bietet E-Assessment mit der M�glichkeit der Verbesserung und Erh�hung der Effizienz im Pr�fungsalltag einen m�glichen Ausweg. Unter E-Assessment versteht man das Nutzen von Hardware und Software im Pr�fungsprozess. So wurden im deutschsprachigem Raum f�r E-Assessment Synonyme wie Online-Pr�fungen oder Computergest�tzte Pr�fungen verwendet \cite[S. 11 ff.]{Ruedel2010}.
Auf dieser Basis entwickelte der Lehrstuhl Praktische Informatik in der Wirtschaft der WWU M�nster das E-Assessment-Tool EASy-DSBuilder zum Assessment von AVL-Baum-Datenstrukturen f�r die Moodledistribution  Learnweb \cite{Usener2014}. Bei Moodle handelt es sich um ein weltweit anerkanntes Lernmanagement-System \cite[S. 33]{Gertrsch2007}, das Lehrenden, Administratoren und Lernenden eine robuste, sichere und integrierte Plattform bereitstellen soll \cite{moo15d}.
%Insbesondere viele bekannte Lernplattformen unterst�tzen den �bungsbetrieb durch Funktionen zu Bereitstellung von Aufgabenbl�ttern und zur Organisation studentischer L�sungen. Einige Systeme bieten dar�ber hinaus eine Elektronische �berpr�fung der L�sung an. \cite{Kuchen2010}.
%Unter Lernplattformen ist Moodle international die am weitesten verbreitete Lernplattform \cite{hei15}. 
%Sie ist weltweit in 231 L�ndern �ber 53.000 Seiten registriert \cite{moo15a}
%Bei Moodle handelt es sich um ein Softwarepaket, welches einen  konstruktivistischen Lehr- und Lernansatz unterst�tzt. \cite{moo15a} 
Weiterhin ist Moodle eine frei verf�gbare Open Source Software. %Dies bedeutet, dass sie frei erh�ltlich ist. 
Au�erdem bietet Sie die M�glichkeit der individuellen Anpassung an spezifische Anwendungssituationen \cite{moo15a}.
%Auch die Westf�lische Wilhelms-Universit�t M�nster stellt zur Verbesserung des Lehrbetriebs eine Moodledistribution unter dem Namen Learnweb zur Verf�gung.
%F�r die Vorlesung \textit{Informatik I} wurde bereits ein Moodlemodul implementiert, welches die M�glichkeit  bietet ????

Im Zentrum dieser Arbeit steht das Moodlemodul EASy-DSBulder. Insbesondere stellt diese Arbeit die Erweiterung dieses Moduls um die Funktionalit�t des Assessments von B-Baum-Datenstrukturen vor.
Hierzu wird die Arbeit durch ein Grundlagenkapitel (Kapitel \ref{sec:grundlagen}) eingeleitet, in dem zuerst die wesentlichen Merkmale der Lernplattform Moodle vorgestellt werden (Kapitel \ref{sec:moodle}). Der anschlie�ende Teil der Grundlagen ist eine Einleitung in die Grundlagen des E-Assessements (\ref{sec:eassessement}). Das Grundlagenkapitel wird durch die Definition der B-Baum-Datenstruktur abgeschlossen(Kapitel \ref{sec:bbaum}). Das darauf folgende Kapitel \ref{sec:dsbuilder} stellt das Moodlemodul EASy-DSBuilder vor. Hierbei wird insbesondere auf die Funktionalit�t aus Benutzersicht und auf die Struktur aus technischer Sicht eingegangen. Im Kapitel \ref{sec:anforderungen} werden die Anforderungen an ein Tool zum Assessment von B-Baum-Datenstrukturen vorgestellt. Insbesondere werden im Kontext des EASy-DSBuilder �nderungsanforderung an dieses Modul zur Umsetzung des Assessments von B-Baum-Datenstrukturen dargelegt. Anschie�end wird die Umsetzung der Anforderungen betrachtet. Hierbei werden die entwickelten Funktionalit�ten beleuchtet und es wird ein Einblick auf ausgew�hlte Implementierungsdetails gegeben. Abschie�end wir ein Fazit des Ergebnisses dieser Arbeit betrachtet und ein Ausblick auf m�gliche weitere Entwicklungen des Moduls gegeben.

