\section{Einleitung}
\label{sec:einleitung}

Bei Moodle handelt es sich um ein Softwarepaket, welches einen  konstruktivistischen Lehr- und Lernansatz unterst�tzt. \cite{moo15a}
 Weltweit in 231 L�ndern �ber 53.000 Seiten registriert \cite{moo15a}
 
 Moodle ist international die am weitesten verbreitete Lernplattform \cite{hei15}.
 
%Moodle ist eine frei verf�gbare Open Source Software. Dies bedeutet, dass sie frei erhaeltlich ist. Au�erdem bietet Sie die M�glichkeit der individuellen Anpassung an spezifische Anwendungssituationen \cite{moo15a}.

Die Westf�lische Wilhelms-Universit�t M�nster stellt zur Verbesserung des Lehrbetriebs eine Moodledistribution unter dem Namen Learnweb zur Verf�gung.

F�r die Vorlesung \textit{Informatik I} wurde bereits ein Moodlemodul implementiert, welches die M�glichkeit  bietet ????

Die Arbeit wird durch ein Grundlagenkapitel (Kapitel \ref{sec:grundlagen}) eingeleitet, in die zentralen Ideen von E-Assessment vorgestellt und 
die wesentlichen Merkmale der Lernplattform Moodle hervorgehoben werden. Bei der Vorstellung von Moodle wird auf die Pluginstruktur der Plattform eingegangen. 

Im darauffolgenden Kapitel (Kapitel \ref{sec:dsbuilder}) wird das Modul EASy-DSBuilder vorgestellt. Hierbei wird auf die Funktionalit�t aus Benutzersicht und auf die Struktur aus technischer Sicht eingegangen.

