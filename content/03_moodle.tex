\subsubsection{Aufbau eines Plugins}
\label{sec:aufbauPlugin}

F�r jedes Plugin in Moodle muss eine bestimmte Datenstruktur implementiert werden. Diese besteht aus separaten Unterverzeichnissen und verpflichtenden Dateien. Des weiteren haben Entwickler die M�glichkeit weitere Dateien selbst zu gestalten. 

\paragraph{/backup}

\paragraph{/db}
\begin{itemize} 
	\item \textbf{/access.php} In dieser Datei werden die so genannten \textit{capabilities} f�r das Plugin definiert. \textit{capabilities} beschreiben die Berechtigungen, welche eine Rolle in diesem Plugin zugeordnet bekommt. Eine Berechtigung ist beispielsweise das hinzuf�gen einer neuen Instanz diese Plugins zu einem Kurs.
	\item \textbf{/install.xml}
	\item \textbf{log.php}
	\item \textbf{upgrade.php}
\end{itemize}
\paragraph{/lang}

\paragraph{/}