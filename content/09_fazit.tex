\section{Fazit und Ausblick}

In dieser Arbeit wurde die Erweiterung des Moodle-Moduls EASy-DSBuilders um das Assessment von B-Baum-Datenstrukturen dokumentiert. 
Hierzu wurde eingangs der EASy-DSBuilder in seiner Struktur und Funktionalit�t vorgestellt. Anschlie�end folge eine Anforderungsanalyse zur Erweiterung des Moduls um die Funktionalit�t eines Assessments von B-Baum-Datenstrukturen. 
Darauf folgend wurden die Stellen im Modul vorgestellt, an denen Modifikation und Erweiterungen durchzuf�hren waren, um die auf Basis der Analyse erarbeiteten Anforderungen umsetzen zu k�nnen. Als Ergebnis kann ein funktionsf�higes Modul in seiner zweiten Version pr�sentiert werden, welches das Assessment von B-Baum-Datenstrukturen unterst�tzt. 

So kann der Lehrende in dieser zweiten Version des EASy-DSBuilders bei der Initialisierung einer neuen Instanz dieses Moduls zwischen den zu pr�fenden Datenstrukturen w�hlen. 
Dem Studierenden wurde hingegen ein erweiterter Editor bereitgestellt, der f�r das Assessment von B-B�umen optimiert ist. Dem Studierenden wird die M�glichkeit geboten, die auf dem Editor befindlichen Schl�ssel intuitiv zu bewegen und auf einer B-Baum-Grundstruktur im Hintergrund, welche sich weiterhin automatisch anpasst, abzulegen. Dabei wurden s�mtliche Funktionalit�ten des Moduls der ersten Version beibehalten und auf das neue Assessment �bertragen. 
Hierzu geh�rt die M�glichkeit, im zehnschrittigen Assessment durch die einzelnen Schritte zu navigieren. 
Eine weitere wichtige Funktion ist die des Feedbacks. 
So kann der Lehrende auch f�r das Assessment von B-B�umen ein Feedback einstellen, welches dem Studierenden Informationen dar�ber gibt, welche Fehler er bei seiner Eingabe gemacht hat.


Eine M�glichkeit der Verbesserung des Moduls liegt in der Optimierung der Positionierung der Elemente eines B-Baums im Editor.
In der gegenw�rtigen Version wird jedem Element ein fester Platz zugeordnet. So bleiben auf der horizontalen Achse Freir�ume zwischen einzelnen Elementen, wenn eine Schicht nicht voll belegt ist. 
Auf Grund des exponentiellen Wachstums und der vollen Ausgeglichenheit der H�he braucht der B-Baum insbesondere bei h�herer Ordnung \textit{m} so viel Platz auf der x-Achse, dass die Usability eingeschr�nkt wird.
Hier sind die unn�tigen Zwischenr�ume insbesondere in der Blattschicht zu eliminieren, um den Gesamtbaum zu Verschlanken.